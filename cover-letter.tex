
% % document type % %
\documentclass[11pt]{article}

% % preamble % %
\usepackage{amsmath} % % centers and provides equation numbers for align env
\usepackage{amssymb} % % allows use of normal N symbol
\usepackage{graphicx} % % allows graphics floats
\usepackage{grffile} % % allows more image file names
\usepackage{subcaption} % % allows subfigures in floats
\usepackage[margin=1in]{geometry}
\usepackage[hidelinks]{hyperref} % % allows URLs and in-document hyperlinking
\usepackage{color}
\usepackage{setspace} % % allows line spacing
\usepackage{moreverb} % % allows use of verbatimtab
\renewcommand\verbatimtabsize{4\relax} % % sets verbatimtab indent to half of default, looks better
\usepackage{parskip} % % don't indent new paragraphs
\usepackage{enumerate}
\usepackage{harvard}

% change font to times new roman
\usepackage{fontspec}
% \usepackage{polyglossia} 
% \setmainlanguage{german}
% \setmainfont{Times New Roman}
\setmainfont{texgyretermes-regular.otf}[
  BoldFont=texgyretermes-bold.otf,
  ItalicFont=texgyretermes-italic.otf
]

% % header and footer % %
\usepackage{fancyhdr}
\fancypagestyle{plain}{
	\fancyhead[L]{
            \includegraphics
            [height=60pt] %40pt
            {\InstituteSeal}
            % {cas_logo.png}
        }  % left header
	\fancyhead[R]{\today} % right header
	\fancyfoot[C]{} % % remove default centered page numbers
}

\fancypagestyle{empty}{
	\newgeometry{top=1.1in,bottom=1in,left=1in,right=1in}
	\fancyfoot[C]{} % % remove default centered page numbers
}

% % omit line beneath header
\renewcommand{\headrulewidth}{0pt}

% % expand head of document:
\setlength{\headheight}{44.35004pt}

% % application specific information % %
\usepackage{school}

\renewcommand*\paragraph[1]{}

% % Change font of whole document
% \renewcommand{\familydefault}{\sfdefault}

% % document begins % %
\begin{document}

% % header w/ logo on first page
\thispagestyle{plain}

% % no header or footer on second page
\pagestyle{empty}

% % address block % %
Faculty Search Committee \\
\DepartmentName \\
\SchoolName \\
\InstitutionName \\
\DepartmentAddress

% % body % %
Dear Members of the Search Committee:

I am writing to express my interest in the tenure-track faculty positions in the Computer Science program within the College of Engineering and Science (COES) at Louisiana Tech University. I will complete my Ph.D. degree in Computer Science from Purdue University by July 2023, specializing in Machine Learning and Database Systems, and have a proven record of accomplishment in research and outstanding teaching skills.

% After being informed of this position by a colleague, 
I was motivated to apply 
as I believe I can contribute to \shortInstitutionName{}'s mission of providing 
% accessible and inclusive 
inclusive and quality
education with my commitment to pursue teaching, the experience of attending a multi-faceted public research university, and my focus on impactful research and education. 
% \shortInstitutionName{}'s mission of brining together industry, and academia in dynamic project-based learning and purpose driven research would allow me to work in a truly collaborative and familiar environment.
% to shape the way emerging technologies influence society.
% The interdisciplinary research in \InstitutionName{} with the goal of the common good to alleviate the community and specific focus on Information Retrieval and Data Science would allow me to work in a truly collaborative and familiar environment. \shortInstitutionName{}'s opportunities for the diverse student body and marginalized community, along with the Partner Employment program, bolstered my choice.


\paragraph{2 and 3) Demonstrated ability to teach and interact with both undergraduate and graduate students}
With a teaching 
% and research 
experience of more than six years, I believe I have the necessary skills to fulfill the duties of an \PositionName{} in \DepartmentName{}.
As a Lecturer and teaching assistant in the CS program at Purdue University, I worked in a culturally, ethnically, and economically diverse and ever-evolving environment. This experience strengthened my communication, interpersonal, and organizational skills. Managing large labs with the Undergraduate TAs or handling graduate labs on my own has allowed me to showcase my teamwork and adaptability skills. Teaching students from sophomore to graduate levels required continuous learning and flexibility to facilitate their progression and expectation from the courses. 

%
As a teacher, I aim to develop self-motivation and critical thinking among my students. Recognizing that every student comes from a different background and culture, I always consider the interconnection between the student's current knowledge and learning capability while preparing the lectures. To promote learning by doing, I choose exercises that are both related to the core concepts and challenging so that they have to step outside of their comfort zone and can develop the logical thinking needed for solving research questions and real-life problems. Letting the students choose the modality for lectures and deliverables, giving frequent feedback for homework and projects, grading with a growth mindset, and leaving detailed reasoning are some of my efforts to make them feel autonomous and involved in the learning process. 
% In my graduate classes, the students appreciated using tools like HotCRP or Perusall for teaching literature reviews or paper writing.
Furthermore, teaching post-covid has equipped me with the tools and skills, including course management and automatic-grading
software to teach online or in-person, e.g., Brightspace, Gradescope, or Campuswire.

\paragraph{4) Good communication skills.}
I gained proficiency in delivering talks to large audiences through lectures in labs and classes and frequent presentations in review meetings and conferences with JPL, Northrop Grumman, and DARPA. Being authored and published multiple papers in VLDB, SIGMOD, and IEEE, I am confident in my communication skills. As a full-time lecturer, I had the experience of taking upto 18 credit hours per semester while teaching multiple classes and 
% conducting multiple 
labs on various courses. I have experience teaching programming language courses (Java, C/C++, Python), database courses (SQL), data structure and algorithm, and software engineering courses (JavaScript and PHP) for multiple semesters.
During research, I enjoyed working with our industry colleagues and understood the need for interdisciplinary collaborations between academia and industry. 
% This experience would be helpful for me to integrate with \shortInstitutionName{}'s mission of interdisciplinary collaboration to better the economic, social, cultural, and environmental health of the communities.
% among social justice, the health sector, and academia.

\paragraph{curriculum development for lecture and lab courses in Computer Science/Data Science}
I have prior involvement in developing course materials for \textit{Object-oriented programming} and \textit{Simulation and Modeling}. In addition, during my tenure as a Lecturer, I improved the undergraduate curriculum for \textit{Network Programming} and \textit{Graphics}.
% added to the course curriculum for \textit{Network Programming} and \textit{Graphics}. 
In my prospective career as \PositionName{}, I look forward to working in a collaborative teaching and learning environment with both faculty and students, where I can improve courses incorporating novel teaching  techniques, research, and the most recent theories on learning, such as Self-determination theory, belongingness, etc. 
I am excited to teach CS courses at any levels.
% both undergraduate and graduate levels. 
I would also welcome the opportunity to develop new courses according to the departmental needs.

\paragraph{P4) The ability to contribute through teaching and/or service to the diversity, cultural sensitivity, and excellence of the academic community.}
\paragraph{Diversity}
During classroom communication, I actively handle the diversity in accents by understanding the difference, addressing the speaking, and actively listening in traditional and alternative methods.
I always create a safe and comfortable environment for all my students to learn and share their feedback and questions. In addition, I have actively engaged in promoting diversity and inclusion through international collaborations, mentoring, outreach, and socially impactful research. As a future teaching faculty of \shortInstitutionName{}, I would continue my efforts through inclusive teaching environments, affinity group participation, outreach to younger generations and beyond borders, and involvement with groups advancing diversity. 
% (such as ADVANCE and CMASS).

\paragraph{P3) Relevant industrial experience beneficial to CS/ DS curriculum development and CS/DS capstone project advising}

During my research, I enjoyed working with our industry colleagues and understood the need for interdisciplinary collaborations between academia and industry.
Regarding guiding students in capstone projects for project-based and collaborative learning, I would like to highlight my experience from research in Multimodal Information Retrieval and from teaching in Computer Networks. I have experience with current industrial developments in transformer-based language representation models, multimodal attention networks, data discovery, and object detection models. Moreover, from my recent TA offering, I am familiar with current developments in network and communication, including Mininet, Software-defined networks, P4-language, 5G, etc. In addition, I am familiar with the CRISP-DM methodology for the data science pipeline and have implemented it in use-cases such as \textit{finding Missing Persons} or \textit{analyzing political bias in newspaper articles}. Under my guidance, multiple undergraduates and Master's students worked in teams on the same project for independent research and learned essential skills in developing practical solutions for real-world problems.  
% Mentoring students on capstone projects would allow me to collaborate with other faculty members and industry leaders. 
% % Yongjian Fu Multimodal information retrieval, Satish Kumar Disaster Resilience, Hongkai Yu Autonomous Driving 
% My experience in interdisciplinary research would require minimal effort from them, benefiting both the students and the research teams.


% CRISP-DM breaks the process of data mining into six major phases:[14]
% Business Understanding
% Data Understanding
% Data Preparation
% Modeling
% Evaluation
% Deployment
%%%%%%%%%%%%%%%%%%%%%%%%%%%%%%%%
% Employee Management System
% Food delivery system
% Instagram Mini
% Hotel Management System
% Bank Management System
% Blood Bank Management
% Android E-commerce
% Bus Ticket Management System
% Hospital Appointment 
% Online Library and Bookshop
% CS 536 courses

\paragraph{1 and P1) Earned doctoral degree in computer science or a closely related field}
% \paragraph{1) Bachelor's and Master's degrees in Computer Science}
I have completed my Master's degree requirements with systems, algorithms, and data-science-related courses, along with elective classes regarding current advancements in Database and Machine Learning. Previously, I completed my Bachelor's in Computer Science with a specialization in Pattern Recognition, where I proposed a novel clustering algorithm for irregularly shaped datasets using a single hyperparameter.
For my Ph.D. dissertation, I propose a system for \textit{Multimodal data discovery} to deliver users with the relevant data at the right time. My proposed system includes solutions for processing large-scale multimodal data, feature extraction from perceptual and textual domains, cross-modal relevance matching, and real-time data delivery. I also proposed a framework for measuring domain complexity in distributed perception domains and a weakly supervised model for open-world cross-modal retrieval to accommodate systems with open-world novelty. I aim to continue my research on Open-world AI and Multimodal information retrieval to understand the hidden patterns in ever-increasing influx of cross-modal data.

% \vfill

% I am very excited about the opportunity of joining \InstitutionName{}, 
I would enjoy further discussing the value I can add to this position and \InstitutionName{}. I am sharing my curriculum vitae, unofficial transcripts, and teaching and diversity 
statements with evidence of my teaching experience 
through the website. Letters of reference will be provided separately for your review. I will gladly provide any other supporting materials upon request. Thank you very much for your time and consideration.


Sincerely, 

% \includegraphics[height=40pt]{signature.png}

\Name{} \\
\normalsize  \textnormal{
          Ph.D. Candidate at Department of Computer Science, Purdue University, West Lafayette, IN 47907
        }\\
        \normalsize \textnormal{ % \small
          \href{mailto:ksolaima@purdue.edu}{ksolaima@purdue.edu} ~|~ %
          (+1) 765-775-8230 ~|~ %
          % Ph.D. Candidate @ Purdue Computer Science
          % 305 N University St, LWSN B132, West Lafayette, IN 47907
          \href{https://ksolaiman.github.io/}{https://ksolaiman.github.io/}
          % \ResumeUrl{https://blog.fkynjyq.com}{blog.fkynjyq.com} \footnote{下划线内容包含超链接。},%
          % \ResumeUrl{https://github.com/fky2015}{github.com/fky2015}%
}

% % end % %
\end{document} 
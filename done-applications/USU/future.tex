\noindent\rule{2cm}{4pt}~\textbf{\textsc{Future Research Agenda.}}
Currently, we are experiencing a thrilling era for multimodal information processing and robust AI research since it is highly supported by the core programs in NSF's Division of Information and Intelligent Systems (IIS) and by "Harvesting the Data Revolution ($HDR^2$)" idea - second wave of one of the 10 big ideas
% supported
by NSF for long-term research.
% Division of Information and Intelligent Systems (IIS)

% Looking forward, I plan to continue investigating the interaction of data management systems with cloud computing, modern hardware (e.g., emerging storage and network), and novel applications (e.g., data science and IoT).

My \textbf{long-term goal} is to create intelligent systems that can reason, learn and cooperate with humans to improve the standard of living by utilizing the vast amounts of data available in the modern era. My focus is to devise new algorithms and methods that can make a significant impact on society, leverage existing scientific advancements, and address real-world challenges. To that end, I
plan to continue my research on \textit{multimodal data management in real world} by approaching from the following directions:

\begin{itemize}[leftmargin=0pt, topsep=0pt,itemsep=0em]
    \item \textbf{\textit{User Preference Modeling:}}
To complete the life-cycle of \textit{situational knowledge delivery}, we still have challenges in modeling user's information need in a robust and efficient manner in multiple directions \cite{solaiman2021applying}:
\begin{enumerate*}[label=(\arabic*)]
    \item user requirement is not always obvious or explicitly stated,
    \item user can be interested in multiple types of events and knowledge bases with varying probabilities,
    \item learning algorithms need to \textit{adapt to changing user preferences with time}.
\end{enumerate*}
%%%
% Multiple Data of Interest to Same User - user interested in multiple knowledge base.
 %%%%%
%%%
I aim to develop novel algorithms using techniques such as active learning and reinforcement learning that can accurately capture and predict users' preferences based on their behavior, interactions, and feedback.
Understanding the features that drive user preferences, and leveraging this knowledge to improve personalized recommendations and user experience, has applications in education (student advising, classroom teaching), e-commerce, healthcare, etc.
%
% Builds User Profiles using the history of user queries
% Active Learning to narrow/expand intention model with more interaction
% Expands user queries with word embedding models to fetch relevant data from the database.
%
To achieve this research goal, collaborations with researchers in \textbf{ human-computer interaction, psychology, and marketing} will be essential. 
%%
\item \textbf{\textit{Explainability and Trustworthiness in Data Recommendation.}}
% and information consumption}
As the amount of multimodal data generated and consumed by users is increasing, there is a growing need for users to understand the basis for recommendations \cite{solaiman2022femmir}
% (similar features \cite{solaiman2022femmir}) 
and the saliency and trustworthiness of the information being consumed. This is especially important in sensitive domains such as \textit{healthcare, finance, and legal decision-making} to allow for tracking, cross-checking with social contexts and verification.
% My research aims to develop novel models and techniques that enable users to better understand and trust the data they receive.
To achieve this goal, collaboration across multiple areas is necessary, including\textbf{ data science, 
% machine learning,
natural language processing, computer vision,  human-computer interaction, and ethics}. 
% By bringing together experts from these areas, we can create more transparent and explainable models for data recommendation and consumption. Additionally, 
% With this, we can ensure that these models are designed with the user in mind, taking into account their cognitive and perceptual abilities. 
This collaboration can also lead to the \textit{development of ethical guidelines and principles for designing trustworthy systems}, ensuring that users' rights and privacy are protected.
%%%
%%
\item \textbf{\textit{Privacy preserving Data Dissemination and Federated Learning.}}
To address the growing concern over data privacy, particularly in medical and identity contexts, research in privacy-preserving multimodal data dissemination and federated learning is crucial, as identified in SKOD framework \cite{palacios2019wip}. Further research to integrate the use of local data processing and remote federation with multimodal machine learning techniques is needed to ensure this new requirement in information processing, while understanding and formalizing the resource requirements. 
% Federated learning offers a promising approach for training large multimodal models without the need to share data with other sites, ensuring data privacy while still benefiting from data from other sites. 
Collaboration across various fields such as \textbf{information security, statistics, data management, law, ethics, and public policy} is vital to advance research in this area.

%%%%%% Federated Learning / Data virtualization/data federation
% 5.11 and 5.12 in \href{https://arxiv.org/pdf/2303.06471.pdf}{``mmdintegration for oncology''}.

% \semisection*{Data Democratization}
    % Missing piece of information in one modality can be filled in with similar information from another modality.
    % fill in missing data one modality with another

\item \textbf{\textit{Information Completion and Data Democratization.}}
As data becomes increasingly important in all domains, there is a need for new techniques that enable individuals and organizations to efficiently extract insights from data and complete missing information. To address this challenge, future research should focus on developing advanced machine learning models that are able to perform well even with incomplete data, as well as methods for effective data integration and knowledge transfer within organizations. Collaboration is needed between \textbf{machine learning experts, data management specialists, and domain experts in various fields} to achieve a comprehensive and effective solution for data democratization and information completion.

\end{itemize}

% With advant of IIS, CISE-IIS (info-integration-informatics)https://new.nsf.gov/funding/opportunities/iis-information-integration-informatics-iii
% and IIS-Human-Centered Computing, IIS-Robust Intelligence (RI), and Safe Learning-Enabled Systems
% https://www.nsf.gov/news/special_reports/big_ideas/nsf2026.jsp
% Aidan Zhang: https://www.nsf.gov/awardsearch/showAward?AWD_ID=2008208&HistoricalAwards=false
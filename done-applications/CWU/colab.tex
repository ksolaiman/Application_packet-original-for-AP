\semisection*{Research Program at CWU}
To establish a successful and inclusive research program at a predominantly undergraduate institution (PUI) like CWU, I will adopt the following core principles:

\begin{enumerate}[label=(\arabic*),topsep=0pt, leftmargin=0pt,nosep]
    \item \textbf{Collaborative Interdisciplinary Research:} I will collaborate with colleagues from diverse fields, both within and outside of \shortInstitutionName{} computer science, to explore interdisciplinary research areas. This approach will lead to impactful outcomes and create additional funding opportunities, such as NSF-ROA, NSF-REU, Cottrell Scholars, and AMS Centennial Research Fellowship. For example, my research in \textit{privacy-preserving data dissemination and federated learning} would require experts from mathematics and ethics with investigations into domain complexity analysis and ethical implications of shared data. 

    \item\textbf{Integrating Research with Courseworks:} I will integrate research experiences into thesis-based courses and project-based labs to create innovative teaching models. By providing early research training and offering students a taste of research, these courses will foster a collaborative spirit and enable students to publish papers and/or join my research team. 
    % Continuation of these seminar-style courses will expand research opportunities.
    Attending and mentoring students in multiple seminar courses at Purdue has trained me for this.

    \item \textbf{Modular Research Approaches:} Working with undergraduate student researchers presents unique challenges, such as student recruitment, time-intensive training, and high turnover. Recognizing the diverse interests and backgrounds of undergraduate and master's students, in my previous works, I have broken down large research problems into smaller modules with tangible outputs. For example, the project on \textit{situational knowledge on demand} was broken down into several modules and sub-modules for different teams to work on. I will continue with this approach with realistic, short-term milestones and modular projects, which will allow for targeted engagement and enhance the students' research experience. 

    \item \textbf{Direct Mentorship and Close Engagement:} While PUIs may have limited financial resources, it presents an opportunity for close faculty-student mentorship. The absence of graduate students or post-doctoral researchers will allow undergraduate students to be involved in every step of the research project, leading to a sense of ownership and honing their research skills, from conception to publication. I will enjoy this process because of the direct relationships and the satisfaction of completing a project.
    % The focused mentorship and engagement throughout their development enable undergraduate researchers to graduate with valuable research experience.

    \item \textbf{Early Introduction to Research and Research Lab Design:} By running my research lab like a mini-research group and involving students from their first or second year while keeping the ignited ones until they graduate, the lab's momentum will be maintained. The older students will be able to pass on knowledge to younger students, utilizing their enthusiasm and work ethic. Recognizing that each hour in the lab is significant for students starting with no prior experience, I will be creating appropriate and intriguing research questions for students of each level. I must carefully select projects to ensure publications and grant funding due to the slower pace of research. 

    % \item \textbf{Flexible Timelines:} Considering that students may have limited hours during the school year or work full-time in the lab for a shorter summer experience, early recruitment and planning are crucial for quick integration into the lab. The shorter timescale also allows for experimentation and the flexibility to pivot to new projects in subsequent terms.

    % \item \textbf{Unique Challenges and Strategies:} Working with undergraduate student researchers presents unique challenges, such as student recruitment, time-intensive training, and high turnover. As mentioned before, I must design the research problems with realistic, short-term milestones and modular projects to overcome these challenges effectively.


    % \item \textbf{Nurturing Good Habits:} Recognizing that each hour in the lab is significant for students starting with no prior experience, I will be creating appropriate and intriguing research questions for students of each level.

    \item \textbf{Inclusive and Supportive Environment:} Engaging undergraduate students in my research program will create a scientific training ground that combines theory and practice, fosters inclusivity, and supports students in their scientific pursuits. This is particularly crucial for historically underrepresented students. Intensive training in technical and communication skills will build confidence, produce impactful science and nurture innovative minds.
\end{enumerate}

Overall, my research plan at \shortInstitutionName{}
% a predominantly undergraduate institution 
will involve collaborative and interdisciplinary research, integrating research with teaching, adapting modular approaches, fostering direct mentorship, being resourceful, and creating an inclusive and supportive environment for student researchers.
%
%
\customsection*{Collaboration and Funding}
%
I gained extensive expertise in overseeing and directing major projects, encompassing teams of over 15 individuals and collaborating with various universities and institutions.
I led multiple masters and undergraduate students (7 of them were undergraduates), collaborated with multiple Ph.D. students and coordinated with 5 professors from different universities to participate in the REALM project. I am fortunate to have close collaborations with professors from multiple universities and research institutes, such as Massachusetts Institute of Technology (MIT), University of Michigan (UMichigan), University of Southern California (USC), Information Sciences Institute (ISI), Institute for Defense Analyses (IDA), University of Massachusetts (UMass), Middle East Technical University (METU), etc. I also have had the fortune to work closely with researchers from databases and applications, along with end-users and program managers to conduct interdisciplinary research. 
% I intend to maintain my current collaborations while actively cultivating new partnerships to advance the establishment of robust principles that underpin research in multimodal knowledge and novelty in learning models. 
%
To realize my future research vision, I plan to collaborate with accomplished researchers from various disciplines, such as natural language processing, computer vision, machine learning, data mining, social science, human-computer interaction, systems, and databases. This collaborative effort extends to CWU, where I anticipate collaborating with experts like Szilárd Vajda, Boris Kovalerchuk, and others. 
Additionally, I aim to establish connections with researchers from other predominantly undergraduate institutions and research institutions beyond CWU to advance the establishment of robust principles that underpin research in multimodal knowledge and novelty in learning models. 
% industry partners, govt, graduate-focused institution.

% \semisection*{Research Program at CWU}
% % At our particular institution, a majority of our students are first-generation college students and come from low-income backgrounds in underserved communities. The ability to nurture their growth in a one-on-one mentoring relationship directly contributes to their success in the future, ultimately supporting the longevity of our region.

% To develop a succesful and inclusive research program at CWU, firstly, I need to recognize the plans would be different from main-stream research-focused universities. I would design my research activities focusing on the following core principles:
% \begin{enumerate*}
%     % \item While most PUIs are individually small, there are many of them that together serve many students. With sufficient support, these teaching-intensive schools collectively provide a large potential pool of highly trained individuals ready for graduate school or industry.

%     % \item Transitioning from an R01 institution to a PUI I do not fully expect to continue the research interests pursued during graduate or post-doctoral positions presents an opportunity for personal and scientific transformation; rather than continuing in my niche of optics I will undertake completely new projects that will leverage the strength of interdisciplinary collaboration at CWU.
%     % if needed, rather than continuing in my niche of optics I will undertake completely new projects that will leverage the strength of interdisciplinary collaboration at CWU.
%     \item Collaborate with colleagues outside of computer science to explore truly interdiscipliary research and impactful outcome, and to create more funding opportunities, such as Cottrell Scholars, AMS Centennial Research Fellowship, etc.
%     For example, research in Privacy preserving Data Dissemination and Federated Learning would branch out to include investigations of domain complexity analysis, as well as ethical implication of shared data. 
%     % All of the students involved in the project thus far have matriculated directly into graduate and professional programs or the work force. 
%     \item Combine research with project-based courses to create new teaching models.
%     %
%     What undergraduates lack in experience and knowledge, they more than make up for with determination and excitement to get the job done. To this end, we have created classes with built-in research experiences either within a specific laboratory or as a full research class to both provide early research training and give students a taste of research.
%     %
%     The collaborative spirit fostered in 
%     % my “Using data science to find hidden chemical rules” 
%     these classes led students to write class papers. These papers have either been published or are being pursued further by students who have joined my research team after the class. We hope to continue these research-focused courses to broaden research opportunities.
%     \item Create smaller modules that are amenable to a wide variety of student interests and student backgrounds.
%     % something that would have been more difficult to achieve had I continued in the niche in which I was trained. 
%     % It is rewarding to watch the personal and professional growth of my students as they undertake novel research.
%     Working with undergraduate and masters students has taught me to break down large research problem into smaller modules with tangible outputs. 
    
%     \item While the funding situation limits what we can do financially, it also forces us to become more resourceful. As a result, undergraduate students and faculty mentors develop research projects together without the degrees of separation that one might find at a larger institution.
%     % \item At small institutions, it is easy for students to have a close relationship with their faculty mentor because PUIs typically have a small number of faculty teaching a small number of students. 
%     This close relationship leads to more direct faculty-student mentorship in the lab in part because PUIs rarely have graduate students or post-doctoral researchers to help supervise undergraduate researchers. 
%     %
%     This means that the student is involved in every step of the project, from conception to publication. Many students conduct their projects over the course of multiple years and mature as they develop their projects, having a strong sense of ownership.
%     %
%     So, undergraduate researchers from PUIs typically graduate with research skills honed by very focused mentorship and engagement at all stages of their development. 

%     % For our students, all of the training and skill development is facilitated by the faculty mentor directly. This means that the student is involved in every step of the project, from conception to publication. Many students conduct their projects over the course of multiple years and mature as they develop their projects, having a strong sense of ownership.

%     \item % For this reason, 
%     it is advantageous to run your PUI research group like a mini-research group and bring in students as first-year or second-year students, keeping the ones who are ignited by doing research until they graduate. This early-introduction-to-research model allows the older students to pass on knowledge to the younger students and keeps momentum going in the lab. What undergraduates lack in knowledge, they usually make up for in enthusiasm and work ethic. Because the pace of research is slower, faculty at PUIs have to be careful about the projects they choose if they want to publish papers and have grant proposals funded. % (which we do!).

%     \item many of which are unique to working with undergraduate student researchers. In particular, student recruitment can be especially challenging, as this often involves balancing the disciplinary knowledge and maturity of a student with their year of study. As well, training undergraduate students to be researchers is typically very time intensive and usually requires a significant amount of one-on-one interaction between the professor and student. This time investment becomes even more pronounced considering the fact that most undergraduate student researchers are only employed for a relatively short amount of time before they graduate. Additionally, research progress can be quite slow due to the relatively high turnover of undergraduate student researchers. Designing undergraduate research programs with realistic, attainable, short-term milestones and modular research projects thus becomes another challenge for professors to overcome.


%     \item Students may have a few hours each week during the school year or, if we’re lucky, work full-time in the lab for a ten-week summer experience. Many students are just starting to get comfortable and develop independence as they reach the end of their undergraduate degrees. This is another case where early recruiting and having a plan pays off, allowing quick integration of students into the lab. On the other hand, the shorter timescale gives faculty the freedom to try new things out and pivot to a new project in the next term.

%     \item If a student is starting with no prior experience, then each hour in the lab is a significant part of their research career up to that point. This makes it imperative for mentors to invest some time into designing a research program that sets good habits and to engage the researcher with a question they find interesting.

%     \item Engaging undergraduate students in my research program affords an opportunity for our lab to collectively create a scientific training ground that puts theory into practice, is inclusive, and provides a supportive place for a student to try to fail. This is especially critical for students who have historically not been supported and engaged in science. Often, we do this work over the 8 -- 10 weeks in the summer. Our efforts involve intensive training of students not only in technical skills and safety protocols but building up their knowledge of the field, communication skills, and confidence to use that knowledge with care and rigor. My aim has been to produce impactful science and daring minds at the ground level. This is hard and constant work.
    
% \end{enumerate*}

%
% As such, PUI science faculty need to have a “fire in the belly” for generating new knowledge, and that fire needs regular encouragement, i.e., teaching loads and service expectations must be balanced with regular research engagement in order to maintain momentum and productivity.


During my Ph.D., my work has been mainly supported by the Northrop Grumman Corporation (NGC), 
% Research Consortium for Artificial Intelligence and Machine Learning
DARPA, ARFL, and Sandia National Lab. 
% I have been nominated by DARPA as a DARPA Riser (awarded to young faculties, postdocs, and senior PhDs who will lead to technological breakthroughs), and I presented ideas directly to DARPA project managers during the invitation-only DARPA Forward Event.
Additionally, I have contributed significantly to the writing of grant proposals, including idea
generation, method design, idea illustration and visual aid creation, such as 
% DARPA KAIROS project,
DARPA ITM project, and DARPA Triage Challenge.
% DARPA SemaFor project, DARPA CCU project, and NSF MMLI project. 
For my future proposals at CWU, I will be designing them as clearly articulated and doable projects with clearly formulated hypothesis and research questions, a definitive plan for implementation and clear milestones. 
As a future faculty, I will continue to seek funding opportunities in the future from early career supports, various funding agencies (e.g., DARPA, ARL, AFRL, IARPA, NSF, NIH, DOE, DOD) and industries (e.g., NGC, Microsoft, IBM, Ford, Meta, Google, Intel) with collaborations from recognized experts and colleagues from my research area.
Specifically I will aim for NSF CAREER, CRII, EAGER, and ADVANCE awards, OSR young investigator programs from DOE, DARPA, AFRL, NASA, and other research awards. As faculty at a PUI, I will be using the RUI and ROA designations for the NSF proposals. Besides these, I will be pursuing private funding such as Sigma Xi, America's seed fund, American Summer/Short-Term Research Publication Grant, etc.

% -	Think – Pair – Share
% -	Partnership with recognized experts
% o	A clearly articulated, doable project • State hypothesis or research questions clearly • A well-formulated argument for why this is important, how this fits into previous work, and why you are the one to do it • A definite plan for implementation • What will success look like?


% CISE .nsf.gov

% Ford alliance

% Sandia labs

% AFRL norm Ahmed mark linderman

% NGC Jason Kobe’s

% PLM manufacturing center

% Career award training at Purdue

% Talk some ML and AI for security or other directions\textbf{}
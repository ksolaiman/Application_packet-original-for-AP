For undergraduate students, I would be happy to teach the fundamental courses, including %structured and object-oriented
programming languages, introductory programming, data structures and algorithms, discrete mathematics, theory of computation, data science, and computer graphics. 
As a prospective teacher, I am also excited to offer software engineering, databases, networks, compilers, information retrieval, data management systems, and machine learning courses. 
% to both undergraduate and graduate students. I know the differences between them from my teaching experience at both levels.
% Specifically, for graduate students, I can offer advanced information retrieval, natural language processing, distributed database systems, data science, or data mining.
% compiler, software development, programming language,
% I took graduate level Compiling and Programming Systems in Fall'22, and am familiar with llvm
%
Moreover, I have plans to offer courses related to multimodal information retrieval and adaptable and explainable AI to students with a research interest. 
These courses will be based on recently published papers in top conferences in machine learning, information retrieval, OpenAI, and XAI conferences. The course projects will be designed in a way that can lead to publications and can mimic peer reviewing. 
%

% what courses you would develop new to the curriculum%%%%%
Besides that, I would also like to extend the current curriculum based on my research experience with \textit{``Applied Machine Learning for Open World Systems"}. Tentative topics include data cleaning, handling lack of annotations, scalability, weakly supervised learning, multimodal information retrieval and feature extraction, intrinsic and extrinsic complexity of system domains, and case study of real-world use cases. Since learning algorithms must be adaptive to real-world situations, it will also include detection, adaptation, and difficulty analysis of novelties in machine learning algorithms.

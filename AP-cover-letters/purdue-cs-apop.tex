
% % document type % %
\documentclass[10pt]{article}

% % preamble % %
\usepackage{amsmath} % % centers and provides equation numbers for align env
\usepackage{amssymb} % % allows use of normal N symbol
\usepackage{graphicx} % % allows graphics floats
\usepackage{grffile} % % allows more image file names
\usepackage{subcaption} % % allows subfigures in floats
\usepackage[margin=1in]{geometry}
\usepackage[hidelinks]{hyperref} % % allows URLs and in-document hyperlinking
\usepackage{color}
\usepackage{setspace} % % allows line spacing
\usepackage{moreverb} % % allows use of verbatimtab
\renewcommand\verbatimtabsize{4\relax} % % sets verbatimtab indent to half of default, looks better
\usepackage{parskip} % % don't indent new paragraphs
\usepackage{enumerate}
\usepackage{harvard}

% change font to times new roman
\usepackage{fontspec}
% \usepackage{polyglossia} 
% \setmainlanguage{german}
% \setmainfont{Times New Roman}
\setmainfont{texgyretermes-regular.otf}[
  BoldFont=texgyretermes-bold.otf,
  ItalicFont=texgyretermes-italic.otf
]

% % header and footer % %
\usepackage{fancyhdr}
\fancypagestyle{plain}{
	\fancyhead[L]{
            \includegraphics
            [height=60pt] %40pt
            {\InstituteSeal}
            % {cas_logo.png}
        }  % left header
	\fancyhead[R]{\today} % right header
	\fancyfoot[C]{} % % remove default centered page numbers
}

\fancypagestyle{empty}{
	\newgeometry{top=1.1in,bottom=1in,left=1in,right=1in}
	\fancyfoot[C]{} % % remove default centered page numbers
}

% % omit line beneath header
\renewcommand{\headrulewidth}{0pt}

% % expand head of document:
\setlength{\headheight}{44.35004pt}

% % application specific information % %
\usepackage{school}

\renewcommand*\paragraph[1]{}

% % Change font of whole document
% \renewcommand{\familydefault}{\sfdefault}

% % document begins % %
\begin{document}

% % header w/ logo on first page
\thispagestyle{plain}

% % no header or footer on second page
\pagestyle{empty}

% % address block % %
Faculty Search Committee \\
\DepartmentName \\
% \SchoolName \\
\InstitutionName \\
\DepartmentAddress

\medskip
% % body % %
Dear Members of the Search Committee:

I am writing to express my interest in the position of \PositionName{} at the \DepartmentName{} at Purdue University, as advertised in the job summary. With my passion for teaching, expertise in instructing large undergraduate classes, and commitment to academic excellence, I am confident in my ability to contribute to the Purdue CS community and inspire students to excel in their academic pursuits.

% Throughout my academic and professional journey, I have gained a strong foundation in data science and machine learning, along with proficiency in modern programming languages and operating systems. 
I will complete my Ph.D. degree in Computer Science from Purdue University by July 2023, specializing in Applied Machine Learning and Database Systems, and have a proven record of accomplishment in research and outstanding teaching skills. Previosuly, I have completed my Master's degree requirements with systems, algorithms, and data-science-related courses, along with elective classes regarding current advancements in Database and Machine Learning. My expertise includes leveraging modern data science techniques and toolkits, and I am highly proficient in Python and Java, widely-used programming languages in the field.

In terms of teaching, I am well-versed in creating engaging and effective learning experiences for undergraduate and graduate students. I have experience in developing online content, both synchronous and asynchronous, and adapting it to meet the needs of diverse learners. I am excited to continue contributing to Purdue's renowned online programs, ensuring their continued excellence and scalability in response to increasing enrollment and evolving student requirements. Moreover, I am committed to staying up-to-date with the latest educational techniques, technologies, and industry practices, constantly striving for personal excellence.

% % After being informed of this position by a colleague, 
% I was motivated to apply 
% as I believe I can contribute to \shortInstitutionName{}'s mission of providing 
% % accessible and inclusive 
% inclusive and quality
% education with my commitment to pursue teaching, the experience of attending a multi-faceted public research university, and my focus on impactful research and education. 
% \shortInstitutionName{}'s mission of brining together industry, and academia in dynamic project-based learning and purpose driven research would allow me to work in a truly collaborative and familiar environment.
% to shape the way emerging technologies influence society.
% The interdisciplinary research in \InstitutionName{} with the goal of the common good to alleviate the community and specific focus on Information Retrieval and Data Science would allow me to work in a truly collaborative and familiar environment. \shortInstitutionName{}'s opportunities for the diverse student body and marginalized community, along with the Partner Employment program, bolstered my choice.

\paragraph{You should highlight your experience in teaching and mentoring students at different levels}
As a Lecturer for 2 years and teaching assistant in the CS program at Purdue University for 5 years, I worked in a culturally, ethnically, and economically diverse and ever-evolving environment. This experience strengthened my communication, interpersonal, and organizational skills. 
Leading large labs with the Undergraduate TAs or handling graduate labs on my own has allowed me to showcase my teamwork and adaptability skills. Teaching students from sophomore to graduate levels required continuous learning and flexibility to facilitate their progression and expectation from the courses.
In my graduate classes, the students appreciated using tools like HotCRP or Perusall for teaching literature reviews or paper writing.
Furthermore, teaching post-covid has equipped me with the tools and skills, including course management and automatic-grading software to teach online or in-person, e.g., Brightspace, Gradescope, or Campuswire.
%
\paragraph{Teaching Philosophy}
My teaching philosophy revolves around growing intrinsic motivation, autonomous identity and critical thinking among students, understanding the diversity among students, and leading with a growth mindset.
% As a teacher, I aim to develop self-motivation and critical thinking among my students. % Recognizing that every student comes from a different background and culture and finding the interconnection between the student's current knowledge and learning capability has helped me prepare better lectures. % To promote learning by doing, I choose exercises that are both related to the core concepts and challenging so that they have to step outside of their comfort zone and can develop the logical thinking needed for solving research questions and real-life problems. Letting the students choose the modality for lectures and deliverables, giving frequent feedback for homework and projects, grading with a growth mindset, and leaving detailed reasoning are some of my efforts to make them feel autonomous and involved in the learning process. 

\paragraph{4) Good communication skills.}
I gained proficiency in delivering talks to large audiences through lectures in labs and classes and frequent presentations in review meetings and conferences with JPL, Northrop Grumman, and DARPA. Being authored and published multiple papers in VLDB, SIGMOD, and IEEE, I am confident in my communication skills. 
%
\paragraph{4) Teaching Experience of lot of credits.}
As a full-time lecturer, I had the experience of taking upto 18 credit hours per semester while teaching multiple classes and labs on various courses. I have experience teaching programming language courses (Java, C/C++, Python), database courses (SQL), data structure and algorithm, and software engineering courses (JavaScript and PHP) for multiple semesters. At Purdue, I have instructed for both undergraduate and graduate courses, while working closely with faculties in programming courses and systems programming.

\paragraph{curriculum development for lecture and lab courses in Computer Science/Data Science}
% I have prior involvement in developing course materials for \textit{Object-oriented programming, Databases,} and \textit{Simulation and Modeling}. In addition, during my tenure as a Lecturer, I improved the undergraduate curriculum for \textit{Network Programming} and \textit{Graphics}.
I have prior involvement in developing course syllabi and improving course curriculum for courses such as, \textit{OOP, Databases, Simulation} and \textit{Modeling.}
% In my prospective career as \PositionName{}, 
I look forward to working in the collaborative teaching and learning environment at Purdue with both faculty and students, where I can improve courses incorporating novel teaching  techniques, research discoveries, and the most recent theories on learning, such as self-determination theory, belongingness, etc. 
I am excited to teach courses at both undergraduate and graduate levels. 
I would also welcome the opportunity to develop new courses according to the departmental needs.
During my tenure as a full-time lecturer, my duties included delivering lectures, designing course materials, conducting labs, grading assignments, and providing advising. 
% As a full time lecturer, 
I have also served on dissertation, accreditation and administrative committees.

\paragraph{interdiscipliniary colab}
% During research, I enjoyed working with our industry colleagues and understood the need for interdisciplinary collaborations between academia and industry. 
% This experience would be helpful for me to integrate with \shortInstitutionName{}'s mission of interdisciplinary collaboration to better the economic, social, cultural, and environmental health of the communities.
% among social justice, the health sector, and academia.

\paragraph{P4) The ability to contribute through teaching and/or service to the diversity, cultural sensitivity, and excellence of the academic community.}
\paragraph{Diversity}
I strongly believe in the value of diversity and inclusion and have actively worked to create an inclusive research and teaching environment. During classroom communication, I actively handle the diversity in accents by understanding the difference, addressing the speaking, and actively listening in traditional and alternative methods.
I always create a safe and comfortable environment for all my students to learn and share their feedback and questions. In addition, I have actively engaged in promoting diversity and inclusion through international collaborations, mentoring students from diverse backgrounds, outreach, and socially impactful research. Under my guidance, multiple undergraduates and Master's students worked in teams for independent research and learned essential skills in developing practical solutions for real-world problems. As a future faculty of \shortInstitutionName{}, I would continue my efforts through 
% leading an equal-opportunity research group, 
creation of inclusive environments, affinity group participation, outreach to younger generations and beyond borders, and involvement with groups advancing diversity at \shortInstitutionName{}. 
% (such as ADVANCE and CMASS).

\paragraph{P3) Relevant industrial experience beneficial to CS/ DS curriculum development and CS/DS capstone project advising}
Regarding supervising students, I would like to highlight my experience from research mentoring in Multimodal Information Retrieval and from teaching in Computer Networks. I have experience with current industrial developments in transformer-based language representation models, multimodal attention networks, data discovery, and object detection models. 
Moreover, from my recent TA offering, I am familiar with current developments in network and communication, including Mininet, Software-defined networks, P4-language, 5G, etc. 
In addition, I am familiar with the CRISP-DM methodology for the data science pipeline and have implemented it in use-cases such as \textit{finding Missing Persons} or \textit{analyzing political bias in newspaper articles}. 
% Under my guidance, multiple undergraduates and Master's students worked in teams on the same project for independent research and learned essential skills in developing practical solutions for real-world problems.  
Furthermore, my software development experience in various languages and application areas enhances my ability to integrate theory with practice, preparing students for the challenges of the real world.
% Mentoring students on capstone projects would allow me to collaborate with other faculty members and industry leaders. 
% % Yongjian Fu Multimodal information retrieval, Satish Kumar Disaster Resilience, Hongkai Yu Autonomous Driving 
% My experience in interdisciplinary research would require minimal effort from them, benefiting both the students and the research teams.

% Beyond the classroom, I look forward to collaborating with educators from other colleges within Purdue, such as the College of Engineering, College of Science, and the School of Engineering Education. During my research, I enjoyed working with our industry colleagues and understood the need for interdisciplinary collaborations between academia and industry. The potential for partnerships with industrial and governmental partners for student projects further excites me, as it presents an opportunity to bridge academia and industry, fostering real-world application of knowledge.

% CRISP-DM breaks the process of data mining into six major phases:[14]
% Business Understanding
% Data Understanding
% Data Preparation
% Modeling
% Evaluation
% Deployment
%%%%%%%%%%%%%%%%%%%%%%%%%%%%%%%%
% Employee Management System
% Food delivery system
% Instagram Mini
% Hotel Management System
% Bank Management System
% Blood Bank Management
% Android E-commerce
% Bus Ticket Management System
% Hospital Appointment 
% Online Library and Bookshop
% CS 536 courses


\paragraph{1 and P1) Earned doctoral degree in computer science or a closely related field}
% \paragraph{1) Bachelor's and Master's degrees in Computer Science}
% I have completed my Master's degree requirements with systems, algorithms, and data-science-related courses, along with elective classes regarding current advancements in Database and Machine Learning. Previously, I completed my Bachelor's in Computer Science with a specialization in Pattern Recognition, where I proposed a novel clustering algorithm for irregularly shaped datasets using a single hyperparameter.
% For my Ph.D. dissertation, I propose a system for \textit{Multimodal data discovery} to deliver users with the relevant data at the right time. My proposed system includes solutions for processing large-scale multimodal data, feature extraction from perceptual and textual domains, cross-modal relevance matching, and real-time data delivery. I also proposed a framework for measuring domain complexity in distributed perception domains and a weakly supervised model for open-world cross-modal retrieval to accommodate systems with open-world novelty. I aim to continue my research on Open-world AI and Multimodal information retrieval to understand the hidden patterns in ever-increasing influx of cross-modal data.

\paragraph{1 and P1) Earned doctoral degree in computer science or a closely related field}
% \paragraph{1) Bachelor's and Master's degrees in Computer Science}
For my Ph.D, I developed algorithms and systems for the retrieval and recommendation of multimodal information in open-world application domains. My Ph.D. dissertation focuses on real applications of learning algorithms and foundations of artificial intelligence and machine learning for open-world novelties, with an emphasis on large-scale machine learning systems, deep learning, representation learning, planning agents, and feature-centered knowledge accumulation. My work with NGC and DARPA resulted in innovative research programs that made an impact in the field, including novel approaches to \textit{resource-aware data management, novel feature extraction methods, label-independent data integration,} and \textit{addressing open-world novelties}. My work in \textit{quantifying and characterizing novelty in open-world domains} had impacts and collaborations in planning domains and autonomous systems. I have successfully applied my work to address open societal problems, such as missing person search, dataset complexity estimation, and medical triage.
%
% \vfill
My long-term goal is to create intelligent systems that can reason, learn and cooperate with humans to improve the standard of living by 
% utilizing the 
% vast amounts of data available in the modern era 
% modern-era data influx
% and 
addressing real-world challenges.
To achieve this, I plan to continue my research on multimodal data management in open-world by focusing on \textit{user preference modeling, improving explainability and trustworthiness in data recommendation with validation and cross-checking, developing algorithms for privacy-preserving data dissemination, and exploring theories on novelties in learning algorithms}. 
%

% \vfill

Thank you for considering my application. 
I have attached my curriculum vitae, a statement of my teaching philosophy, my diversity and inclusion statement, my research statement, and a list of three references as requested. I will gladly provide any other supporting materials upon request.
I am excited about the opportunity to contribute to Purdue University Computer Science in a new role. I eagerly await the chance to discuss my qualifications and potential contributions further.

\medskip

Sincerely, 

% \includegraphics[height=40pt]{signature.png}

\Name{} \\
\normalsize  \textnormal{
          Ph.D. Candidate at Department of Computer Science, Purdue University, West Lafayette, IN 47907
        }\\
        \normalsize \textnormal{ % \small
          \href{mailto:ksolaima@purdue.edu}{ksolaima@purdue.edu} ~|~ %
          (+1) 765-775-8230 ~|~ %
          % Ph.D. Candidate @ Purdue Computer Science
          % 305 N University St, LWSN B132, West Lafayette, IN 47907
          \href{https://ksolaiman.github.io/}{https://ksolaiman.github.io/}
          % \ResumeUrl{https://blog.fkynjyq.com}{blog.fkynjyq.com} \footnote{下划线内容包含超链接。},%
          % \ResumeUrl{https://github.com/fky2015}{github.com/fky2015}%
}

% % end % %
\end{document} 
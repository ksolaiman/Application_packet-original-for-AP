
% % document type % %
\documentclass[10pt]{article}

% % preamble % %
\usepackage{amsmath} % % centers and provides equation numbers for align env
\usepackage{amssymb} % % allows use of normal N symbol
\usepackage{graphicx} % % allows graphics floats
\usepackage{grffile} % % allows more image file names
\usepackage{subcaption} % % allows subfigures in floats
\usepackage[margin=1in]{geometry}
\usepackage[hidelinks]{hyperref} % % allows URLs and in-document hyperlinking
\usepackage{color}
\usepackage{setspace} % % allows line spacing
\usepackage{moreverb} % % allows use of verbatimtab
\renewcommand\verbatimtabsize{4\relax} % % sets verbatimtab indent to half of default, looks better
\usepackage{parskip} % % don't indent new paragraphs
\usepackage{enumerate}
\usepackage{harvard}

% change font to times new roman
\usepackage{fontspec}
% \usepackage{polyglossia} 
% \setmainlanguage{german}
% \setmainfont{Times New Roman}
\setmainfont{texgyretermes-regular.otf}[
  BoldFont=texgyretermes-bold.otf,
  ItalicFont=texgyretermes-italic.otf
]

% % header and footer % %
\usepackage{fancyhdr}
\fancypagestyle{plain}{
	\fancyhead[L]{
            % \includegraphics
            % [height=40pt] %40pt
            % {\InstituteSeal}
            % % {cas_logo.png}
        }  % left header
	\fancyhead[R]{\today} % right header
	\fancyfoot[C]{} % % remove default centered page numbers
}

\fancypagestyle{empty}{
	\newgeometry{top=1.1in,bottom=1in,left=1in,right=1in}
	\fancyfoot[C]{} % % remove default centered page numbers
}

% % omit line beneath header
\renewcommand{\headrulewidth}{0pt}

% % expand head of document:
\setlength{\headheight}{44.35004pt}

% % application specific information % %
\usepackage{school}
\usepackage{ifthen}

\renewcommand*\paragraph[1]{}

% % Change font of whole document
% \renewcommand{\familydefault}{\sfdefault}

% % document begins % %
\begin{document}

% % header w/ logo on first page
\thispagestyle{plain}

% % no header or footer on second page
\pagestyle{empty}

% % address block % %
Faculty Search Committee \\
\DepartmentName \\
% \SchoolName\\
\InstitutionName \\
\DepartmentAddress\\

% % body % %
Dear Members of the Search Committee:

I am writing to apply for the position of \PositionName{} 
\ifx\DepartmentName\undefined
\else
in 
% Computer Science 
% the Software and Information Systems Department
\DepartmentName{} 
\fi
% in the \SchoolName{}
at \InstitutionName{}%. %(\shortInstitutionName{}), 
%%%%
\ifx\shortInstitutionName\undefined
    \ifx\startDate\undefined
        .
    \else
        , beginning \startDate{}. %, as mentioned on your department website. 
    \fi
\else
    ~(\shortInstitutionName).   \ifx\startDate\undefined
    \else
        , beginning \startDate{}. %, as mentioned on your department website. 
    \fi
\fi
%%%%
I am a doctoral candidate at Purdue University, and I am scheduled to complete my Ph.D. degree requirements, including defending and depositing, by July 2023. I have a track record of external research funding from federal agencies and industry partners such as DARPA, Sandia and NGC. I have a strong interdisciplinary orientation in my work and have collaborated with end-users and experts in database, NLP+CV, information and data science, and industry from MIT, NGC, USC-ISI, DARPA, and others. My research portfolio includes multiple publications in VLDB, SIGMOD, AAAI, and IEEE and systems with social impacts.
With a teaching, mentoring and research experience of more than six years, I believe I have the necessary skills to fulfill the duties of an \PositionName{} for the \InstitutionName{}.
%

\paragraph{1 and P1) current research portfolio, accomplishments, and future research goals}
% \paragraph{1) Bachelor's and Master's degrees in Computer Science}
% I have completed my Master's degree requirements with systems, algorithms, and data-science-related courses, along with elective classes regarding current advancements in Database and Machine Learning. Previously, I completed my Bachelor's in Computer Science with a specialization in Pattern Recognition, where I proposed a novel clustering algorithm for irregularly shaped datasets using a single hyperparameter.
% For my Ph.D. dissertation, I propose a system for \textit{Multimodal data discovery} to deliver users with the relevant data at the right time. My proposed system includes solutions for processing large-scale multimodal data, feature extraction from perceptual and textual domains, cross-modal relevance matching, and real-time data delivery. I also proposed a framework for measuring domain complexity in distributed perception domains and a weakly supervised model for open-world cross-modal retrieval to accommodate systems with open-world novelty. I aim to continue my research on Open-world AI and Multimodal information retrieval to understand the hidden patterns in ever-increasing influx of cross-modal data.

%%%%% Just rudimentary level past research detail, didnot connect with application %%%
% I have worked on developing algorithms and robust systems to extract meaningful information from heterogeneous and changing datasets, regardless of whether presented as text, images, videos, audio, or other modalities. % , while achieving data-driven and resource-aware data management and data integration capabilities. 
% I also proposed scientific principles to \textit{quantify and characterize novelty in open-world domains, while creating scalable and efficient AI systems that react to novelty in those domains}. % My research focused on three main challenges for multimodal information processing in open-world: \textit{resource-aware data management system, label-independent data integration, and dealing with open-world novelties}. I have developed several approaches to address these challenges, including SKOD for situational knowledge deliver, EARS for data integration using schema mapping, FemmIR for approximate matching of co-ordinated graph representations of data samples, HART for attribute extraction from text, and WesJeM for embedding data-objects and their semantic properties in a high-dimensional space using higher-level semantic features. My work has had a positive impact on open societal problems, including Missing Person Search, Dataset Complexity Estimation, and Medical Triage. 

I have experience developing algorithms and systems for open-world application domains and retrieval and recommendation of multimodal data. 
My research focused on real applications of learning machines and foundations of artificial intelligence and machine learning for \textit{open-world novelties}, with an emphasis on large-scale machine learning systems, deep learning, representation learning, reinforcement learning, and explanability.
% I also proposed scientific principles to \textit{quantify and characterize novelty in open-world domains, while creating scalable and efficient AI systems that react to novelty in those domains}.
% Research in Applications for Learning Machines (REALM)
% Science of Artificial Intelligence and Learning for Open-world Novelty 
My work with NGC and DARPA resulted in innovative research programs that made an impact in the field, including novel approaches to \textit{resource-aware data management, label-independent data integration,} and \textit{addressing open-world novelties}.
My work in \textit{quantifying and characterizing novelty in open-world domains} had impacts and collaborations in planning domains and autonomous system design.
% 
I have applied my work to address open societal problems, such as missing person search, dataset complexity estimation, and medical triage.

My long-term goal is to create intelligent systems that can reason, learn and cooperate with humans to improve the standard of living by utilizing the 
% vast amounts of data available in the modern era 
modern-era data influx
and address real-world challenges.
To achieve this, I plan to continue my research on multimodal data management in open-world by focusing on \textit{user preference modeling, improving explainability and trustworthiness in data recommendation with validation and cross-checking, developing algorithms for privacy preserving data dissemeniation, and exploring theories on novelties in learning algorithms}. 
% You believe your research can make a significant impact on society by leveraging existing scientific advancements and addressing real-world challenges.
%%%%%%%%%%
% In pursuing my research goals, I would seek to collaborate with other faculty members within the department and across the university who share my interests, including experts in NLP, computer vision (), ethics, psychology, marketing, social science, human computer interaction (), novelties and robust systems (), and data management systems and databases (). 
As I strive to achieve my research objectives, my intention is to foster partnerships with like-minded faculty members at \shortInstitutionName{} and outside, for example, Li Yang, Ajay Gupta, Guan Yue Hong, Alvis Fong, and others. %and other institutions beyond \shortInstitutionName{}. 
%
More information is included in the research statement under `collaboration and funding'.
%

%%%%%%%%%%%%%%%%%%%%%%%%
Additionally, I have made significant contributions to writing grant proposals, including notable projects such as the DARPA ITM project and the DARPA Triage Challenge. My involvement encompassed various aspects, such as generating ideas, designing methods, illustrating concepts, and creating visual aids. For future funding endeavors, I intend to develop research proposals that are well-articulated, feasible projects, with clearly defined hypotheses, research questions, implementation plans, and identifiable milestones. As a future faculty member, I will actively pursue funding opportunities from a range of sources, including early career support, esteemed funding agencies (such as DARPA, ARL, AFRL, NSF, DOE, DOD), and industry partners (such as NGC, Microsoft, IBM, Ford, Meta, Google, Intel), while collaborating with recognized experts and colleagues in my research field. 

% I am excited about the prospect of contributing to UW-Madison's world-class core of interdisciplinary strength in the areas of computational and data-driven science and engineering.


\paragraph{2 and 3) Demonstrated ability to teach and interact with both undergraduate and graduate students}
As a Lecturer for 2 years and teaching assistant in the CS program at Purdue University for 5 years, I worked in a culturally, ethnically, and economically diverse and ever-evolving environment. This experience strengthened my communication, interpersonal, and organizational skills. 
Leading large labs with the Undergraduate TAs or handling graduate labs on my own has allowed me to showcase my teamwork and adaptability skills. Teaching students from sophomore to graduate levels required continuous learning and flexibility to facilitate their progression and expectation from the courses.
%
% \paragraph{Teaching Philosophy}
My teaching philosophy revolves around growing intrinsic motivation, autonomous identity and critical thinking among students, understanding the diversity among students, and leading with a growth mindset. 
% I have actively engaged in promoting diversity and inclusion through international collaborations, mentoring students from diverse backgrounds, outreach, and socially impactful research.
%
% I value accountability and take full responsibility for the quality of my work, promptly addressing any mistakes or concerns in my research and other academic responsibilities. I make informed decisions by carefully analyzing options and determining the best course of action, allowing me to meet deadlines and manage multiple responsibilities effectively. In teamwork, I promote cooperation, mutual support, and active participation, as evidenced by my collaborations with five professors, over 15 students, and multiple industry partners during my Ph.D. These qualities shape my research and teaching approach, fostering a productive and collaborative environment.

% I will feel very familiar to work in a collaborative teaching and learning environment, such as USU, alongside faculty and students. This environment  
I have prior involvement in developing course materials and improving course curriculum for courses such as, OOP, Databases, Simulation and Modeling.
I have experience of working in a collaborative teaching and learning environment at Purdue.
I look forward to teach and enhance courses at \InstitutionName{} by integrating innovative teaching techniques, research findings, and the latest theories on learning, such as self-determination theory and belongingness. I am excited to teach 
\ifx\degreeLevel\undefined 
undergraduate
\fi
courses at 
\ifx\degreeLevel\undefined 
both upper and lower levels.
\else
both undergraduate and graduate levels.
\fi
I would also welcome the opportunity to develop new courses according to the departmental needs.

\paragraph{4) Good communication skills.}
I gained proficiency in delivering talks to large audiences through lectures in labs and classes and frequent presentations in review meetings and conferences with JPL, Northrop Grumman, and DARPA. Being authored and published multiple papers in VLDB, SIGMOD, and IEEE, I am confident in my communication skills. During my tenure as a full-time lecturer, my duties included delivering lectures, designing course materials, conducting labs, grading assignments, and providing advising. I have taught upper and lower-level undergraduate courses for multiple semesters during this time, while I also instructed in graduate-level courses at Purdue University. As a full time lecturer, I have served on dissertation, accreditation and administrative committees.


% Demonstrated commitment to mentoring/advising a multicultural population to expand and enhance their educational experience and career goals within and outside traditional classroom settings in accordance with the university philosophy supporting diversity, equity, inclusion, and academic success}
I have had the privilege of mentoring students from diverse backgrounds, both in terms of their geographical locations, including the USA, Czech Republic, Turkey, and India, and the universities they attended such as MIT, Purdue, and METU. Among them, two students became interested in my research after attending my guest lecture in my advisor's class. Additionally, two undergraduate students from Turkey who I advised went on to pursue graduate programs in Germany. In addition to my own advisees, I have provided guidance and support to number of students from underrepresented groups (URG) and my alma mater in their applications to graduate schools. This support involved revising their application materials, suggesting research directions, and helping them for interviews, among other aspects. 
% Moving forward, I plan to continue mentoring students across borders, as I find great joy in witnessing their success and seek to give back to the altruistic individuals I encountered during my journey as a first-generation international student.


\paragraph{P3) Relevant industrial experience beneficial to CS/ DS curriculum development and CS/DS capstone project advising}

% % During my research, I enjoyed working with our industry colleagues and understood the need for interdisciplinary collaborations between academia and industry.
% Regarding guiding students in capstone projects for project-based and collaborative learning, I would like to highlight my experience from research in Multimodal Information Retrieval and from teaching in Computer Networks. I have experience with current industrial developments in transformer-based language representation models, multimodal attention networks, data discovery, and object detection models. Moreover, from my recent TA offering, I am familiar with current developments in network and communication, including Mininet, Software-defined networks, P4-language, 5G, etc. In addition, I am familiar with the CRISP-DM methodology for the data science pipeline and have implemented it in use-cases such as \textit{finding Missing Persons} or \textit{analyzing political bias in newspaper articles}. 
% I have completed my Master's degree requirements with systems, algorithms, and data-science-related courses, along with elective classes regarding current advancements in Database and Machine Learning. 
% Previously, I completed my Bachelor's in Computer Science with a specialization in Pattern Recognition, where I proposed a novel clustering algorithm for irregularly shaped datasets using a single hyperparameter.
% Under my guidance, multiple undergraduates and Master's students worked in teams on the same project for independent research and learned essential skills in developing practical solutions for real-world problems.  
% Mentoring students on capstone projects would allow me to collaborate with other faculty members and industry leaders. 
% % Yongjian Fu Multimodal information retrieval, Satish Kumar Disaster Resilience, Hongkai Yu Autonomous Driving 
% My experience in interdisciplinary research would require minimal effort from them, benefiting both the students and the research teams.


% CRISP-DM breaks the process of data mining into six major phases:[14]
% Business Understanding
% Data Understanding
% Data Preparation
% Modeling
% Evaluation
% Deployment
%%%%%%%%%%%%%%%%%%%%%%%%%%%%%%%%
% Employee Management System
% Food delivery system
% Instagram Mini
% Hotel Management System
% Bank Management System
% Blood Bank Management
% Android E-commerce
% Bus Ticket Management System
% Hospital Appointment 
% Online Library and Bookshop
% CS 536 courses

% I am very excited about the opportunity of joining \InstitutionName{}, 
I would enjoy further discussing the value I can add to this position and \InstitutionName{}. I am sharing my 
% resume,
curriculum vitae, 
references,
% research, teaching 
% and diversity 
and statements
% and teaching demonstration video
% statement
% with evidence of my teaching experience 
through the website. 
% Letters of reference will be provided separately for your review. 
I will gladly provide any other supporting materials upon request. Thank you very much for your time and consideration.\\



Sincerely, 
% \includegraphics[height=40pt]{signature.png}

\Name{} \\
\normalsize  \textnormal{
          Ph.D. Candidate at Department of Computer Science, Purdue University, West Lafayette, IN 47907
        }\\
        \normalsize \textnormal{ % \small
          \href{mailto:ksolaima@purdue.edu}{ksolaima@purdue.edu} ~|~ %
          (+1) 765-775-8230 ~|~ %
          % Ph.D. Candidate @ Purdue Computer Science
          % 305 N University St, LWSN B132, West Lafayette, IN 47907
          \href{https://ksolaiman.github.io/}{https://ksolaiman.github.io/}
}

% % end % %
\end{document} 
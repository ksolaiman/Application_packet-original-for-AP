\documentclass[10pt]{article}
% \documentclass[11pt,a4paper,ragged2e,withhyper]{altacv}
\usepackage[T1]{fontenc}
\usepackage{mathptmx} %{tgbonum}
% \usepackage{natbib}
\usepackage[inline]{enumitem}
%Options: Sonny, Lenny, Glenn, Conny, Rejne, Bjarne, Bjornstrup
% \usepackage[Sonny]{fncychap}

% Set page margins
\usepackage[margin=1in,top=0.8in]{geometry} % 0.4

% Mathematical things
\setlength{\parskip}{2pt plus 1pt minus 1pt}

% \makeatletter
% \def\paragraph{\@startsection{paragraph}{4}%
%   \z@\z@{-\fontdimen 2 \font}%
%   {\normalfont\bfseries}}
% \makeatother

\makeatletter
\def \section {%
    \@startsection {section}
    {1}%
    {\z@}%
    {-3.3ex \@plus -1ex \@minus -.2ex}%
    % {2.2ex \@plus.2ex}%
    % {-1em}%
    {0.2em}
    {\normalfont \Large \scshape \bfseries} % \Large
    }
\makeatother

% https://latexref.xyz/_005c_0040startsection.html
\makeatletter
\def \paragraph {%
    \@startsection{paragraph}% name
        {4}%  level
        \z@\z@{-\fontdimen 6 \font}%
        % {0pt}% indent 
        % {\normalfont\bfseries}}
        % {\normalfont\scshape\bfseries}}
        % {2pt} %afterskip
        {\large \scshape \bfseries}% style
    }
\makeatother

\newcommand*\bfheading[1]{{\scshape\textbf{#1.}}}

\newcommand*\itheading[1]{\textbf{\textit{#1.}}}

\newcommand*\heading[1]{\textbf{\large #1.}}

\makeatletter
\def \customsection {%
    % \@startsection {section}
    \@startsection{section}
    {1}%
    {\z@}%
    % {-3.3ex \@plus -1ex \@minus -.2ex}%
    % {2.2ex \@plus.2ex}%
    % {-1em}%
    {0.5em}
    {0.5em}
    {\noindent \rule{2cm}{4pt} \large 
    \scshape 
    \bfseries} % \Large \normalfont 
    }
\makeatother

\makeatletter
\def \semisection {%
    % \@startsection {section}
    \@startsection{section}
    {1}%
    {\z@}%
    % {-3.3ex \@plus -1ex \@minus -.2ex}%
    % {2.2ex \@plus.2ex}%
    % {-1em}%
    {0.5em}
    {0.2em}
    {\noindent \normalfont \large \it \bfseries} % \Large
    }
\makeatother

% \makeatletter
% \def \customsection {%
%     % \@startsection {section}
%     \@startsection{section}
%     {1}%
%     {\z@}%
%     % {-3.3ex \@plus -1ex \@minus -.2ex}%
%     % {2.2ex \@plus.2ex}%
%     % {-1em}%
%     {0.1em}
%     {0.2em}
%     {\noindent \rule{2cm}{4pt} \large 
%     \scshape 
%     \bfseries} % \Large \normalfont 
%     }
% \makeatother

% \makeatletter
% \def \semisection{%
%     % \@startsection {section}
%     \@startsection{section}
%     {2}%
%     {\z@}%
%     % {-3.3ex \@plus -1ex \@minus -.2ex}%
%     % {2.2ex \@plus.2ex}%
%     % {-1em}%
%     {0.1em}
%     {0.1em}
%     {\noindent \large  \it \bfseries} % \Large \normalfont 
%     }
% \makeatother

\usepackage[hidelinks]{hyperref} % % allows URLs and in-document hyperlinking

% % header and footer % %
\usepackage{fancyhdr}
\fancypagestyle{plain}{
	\fancyhead[L]{\textit{\InstitutionName}}  % left header     %
 % \href{https://ksolaiman.github.io/}{Website}
 % pg number in footer
	\fancyhead[R]{\email{ksolaima@purdue.edu}} % right header
        \fancyhead[C]{\thepage}
        % \fancyhead[C]{\Name}
	
	\fancyfoot[R]{} % % pg number in footer
	\fancyfoot[C]{} % % remove default centered page numbers
	\fancyfoot[L]{} % last compiled in footer
}

% % application specific information % %
\usepackage{school}


\title{
    \vspace{-3em}
    \textbf{Research Statement} \hfill \href{https://ksolaiman.github.io/}{\textit{\Name}}
    \vspace{-2.5em}
}

% \author{KMA Solaiman\vspace{-2em}}
% \email{ksolaima@purdue.edu}
\date{}

\begin{document}
\maketitle
\pagestyle{plain}

%%
% For long rs, comment out the newcommand in next line. Uncomment the philosophy and detailed-prolog. 
% Uncomment previous def of custom and semisection, and comment the current def.
%%
\renewcommand{\medskip}{\smallskip}

% % \bfheading{Talk Title} Multimodal Information Retrieval in Open-world Environment: Right Information at the Right Time
\noindent \semisection*{Talk Title} Situational Knowledge on Demand in Open-world Environment: Right Information at the Right Time

\noindent \semisection*{Abstract}
Real-world use cases in data-centric disciplines often have an unprecedented influx of unstructured and noisy data from multiple sources and modalities. Extraction of meaningful information from such overwhelming and dirty datasets requires achieving the complementary goals of computational efficiency and contextually-aware multimodal information recommendation. My research aims to understand how these goals can be achieved by designing interactive algorithms and frameworks for scalable and efficient systems that can adapt to changing data patterns and operate under resource constraints in open-world. Such a design poses a few significant challenges: (1) resource-aware data management (involving processes such as data ingestion and delivery, knowledge base formation, and multimodal feature extraction): I will present a novel framework, SKOD, a scalable and on-demand Situational Knowledge Query Engine, to deliver multimodal content to appropriate users through triggers by continuously processing and building a multi-modal relational knowledge base using SQL queries entity-centric concept extraction (such as person – gender, race, clothes, color of clothes) from multiple modalities; (2) label-independent data integration: I will introduce FemmIR, a weakly supervised multimodal retrieval model using Graph Edit Distance to tackle the lack of training samples that can integrate new modalities and produce rankings of exactly and approximately relevant data. Additionally, I will introduce WesJeM, another weakly supervised model that performs a feature-level late fusion using the entity-centric high-level features with semantic concepts and by building a common multimodal representation space; (3) adaption to open-world novelties: After introducing changing data patterns and characterizing ‘novelty’ in multimodal information retrieval, I would discuss how WesJeM tackles novel situations including zero-shot learning. Further, I would explain how we used a novelty generation framework to augment visual modality datasets in a domain-agnostic and budget-efficient manner and improved the object-detection models’ adaptation capabilities. Finally, I will discuss how we learned and leveraged inherent data complexity of visual modalities in different learning environments to identify the difficulty of handling novelty. Such an Intelligent Multimodal and Situation-aware Recommendation Engine for real-world environments pave the path for the next generation information access for many open-ended application domains. I will also show SKOD’s positive results on open-world applications, such as finding missing persons, data discovery, and urban information systems.
 
\noindent \semisection*{Bio}
KMA Solaiman is a Ph.D. candidate in the Department of Computer Science at Purdue University, advised by Prof. Bharat Bhargava and Prof. Michael Stonebraker from MIT. His work was supported by the REALM consortium from Northrop Grumman Corporation and the SAIL-ON project from DARPA. Solaiman has collaborated with West Lafayette Police Department, USC-ISI (Information Sciences Institute), IDA (Institute for Defense Analyses), MIT, and METU. His research investigated the challenges of Multimodal Information Extraction and Situation-aware data recommendation in real-world applications, along with novelties in learning algorithms. He has mentored more than 12 undergraduate and Master’s students. His work has contributed to building systems with societal impacts such as finding missing persons. His work has been published at top machine learning and database conferences and journals such as IEEE Computer, Artificial Intelligence, VLDB, SIGMOD, AAAI, and IEEE TransAI. Additional information is available at https://ksolaiman.github.io/.
% \bfheading
\semisection*{Research interest} Multimodal Information Retrieval, Data Integration, Recommendation Engine, Novelties in Learning Algorithms, Representation Learning, and Data Management Systems.\\
% Data Mining.\\
\noindent \rule{\linewidth}{2pt} 

\newpage
% \bfheading{Research Vision and Research Goal}

Real-world use-cases in data-centric applications 
%% adding from detailed
(including societal, healthcare, or education) with minimal computational resources % ending add
often have an unprecedented influx of unstructured and noisy data from multiple sources and modalities. 
Extraction of meaningful information % refers to recommendation
from such 
% overwhelming and dirty (heteregenous) 
heterogenous and changing % refers to mmir and novelty
datasets requires achieving the complementary 
% goals of
functionalities of 
% finding solutions for the challenges of 
% computational efficiency and contextually-aware multimodal information recommendation.
\textit{
cross-modal matching,
% multimodal information (/data) retrieval,
%%%%
% adaptable learning models, % and systems (for the intric dataset complexity work)
scalable data management system (mostly search engines and databases),
%%%%
}and \textit{situationally-aware data recommendation.}
%%%%% Same data can be interpreted differently by different users based on the context.
%%%%%%%%%%%%%%%%%%%%%%%%%%%%%%%%%%%%%%%%%%%%
% , data management, data integration and relevance, situational awareness, and user modeling.
My research goal is to understand how these
functionalities can be achieved by % goals -> functionalities
designing interactive algorithms and 
robust systems %techniques 
that solve the problem of 
\textbf{situational knowledge on demand in open-world}
% on-demand multimodal information recommendation % retrieval 
from multiple forms of information, regardless of whether
presented as text, images, videos, audio, or other modalities,
while achieving data-driven and resource-aware \textit{data management} and \textit{data integration} capabilities.  \newline \indent
%%%%%%%%%%%%%
%%%%%%%%%%%%%%%
%%%%%%%%%%%%%%%%%
%%%%%%%%%%%%%%%%%%%
My later works branched into developing 
scientific principles to \textit{quantify} and \textit{characterize novelty} (significant and unexpected events) \textit{in open-world domains}, while creating scalable and efficient AI systems that \textit{react to novelty }in those domains.
% scalable and efficient systems that can \textit{adapt to changing data patterns in open-world environment} - 
% \begin{enumerate*}[label=(\arabic*)]
%     \item Novelties in Domain or Data level,
%     \item Novelties in Semantic Concepts,
%     \item Novelties in Planning domain Environments.
% \end{enumerate*}

% My later research morphed into building robust and adaptable data management (includes the domain complexity for perception domain) and data integration systems mainly for perception domain (work w/ Alina focuses on perception domain).
\medskip
Developing intelligent AI systems including situational knowledge recommenders for open-world environments involves tackling multiple \textit{system design challenges}:
% \newline
\begin{enumerate*}[label=\\\textbf{(\arabic*)}]%leftmargin=0.5pt
    \item 
    % \textbf{Resource-aware Data Management:}
% Obtaining optimal performance in open-world multimodal applications requires designing Data Management System that can handle domain-specific workloads, low computational resources, and changing requirements. 
% During our collaboration with local police department and MIT for building Missing-person Query engine \cite{solaiman2021applying} and Human-in-the-Loop Video Querying system \cite{stonebraker2020surveillance}, we built a multimodal knowledge querying system \cite{palacios2019wip} by leveraging the capabilities of the feature extraction models and the functionalities of RDBMS.
% %%
% % higher-level 
% For entity-cetric semantic concept extraction from unstructured text, I proposed a candidate sentence identification and semantic attribute understanding model from unstructured texts \cite{solaiman2022femmir}.
% %%
% For video and image, we proposed priority-polling \cite{stonebraker2020surveillance} for large scale object and attribute detection models in video and images.
% % Will not mention color-sampling in intro-detail, but will mention in long detail. 
%
%
%%%%%%%%%%%%%%%%%%%%%%%%%
%%%%%%%%%% Junagaou style %%%%%%%%
% \textbf{Multimodal Data Preparation: } %for Situational Knowledge on Demand
% Multimodal data preparation, data completion and feature extraction
\textbf{Multimodal Data Preparation, Data Completion, and Feature Extraction.}
For data preparation and information recommendation in modern multimodal applications, we proposed a novel multimodal knowledge querying framework % system
called \textbf{SKOD} (Situational Knowledge on Demand) \cite{palacios2019wip, stonebraker2020surveillance}. SKOD addressed the challenges imposed by heteregeneous restful and streaming input, increasing data volume, delivery-on-demand, lack of computational resources, and changing information needs in emerging multimodal applications, by leveraging the entity-centric higher-level semantic concepts and the functionalities of RDBMS. For
satisfying an incomplete information need over time, we used the Trigger functionality from Postgres after feature extraction and data integration.
% to query domain-specific information needs in real-world use cases. 
%%%%%% 
% Higher-level semantic concepts can include objects, object types, physical relations, e.g., a person wearing blue shirt, time and place of an incident.
% Data management systems require a comprehensive understanding of the data properties, user requirements, and limitations imposed by open-world to achieve optimal performance and scalability for multimodal data recommendation. 
% Emerging multimodal applications impose challenges for traditional systems 
% % (hardware and software) design choices 
% ranging all the way from input 
% % (heterogeneous sources, context and modalities) 
% and output, 
% % (delivery-on-time, quick throughput and diverse users), 
% changing information needs, 
% % (knowledge base creation and query), 
% to computational resources. 
% (lack of annotations, domain-specific feature extractors or human resources). 
% For instance, we showed that transfer learning for fine-grained semantic concept extraction from videos turned out to be ill-suited in large-scale systems \cite{stonebraker2020surveillance}. 
%
%
To tackle the challenges of lack of annotations or task-specific feature extractors for unstructured texts, I proposed an attribute detection model for text for the first time, \textbf{HART} \cite{solaiman2022femmir} using language representation models and syntactic properties of text. We also proposed customizations \cite{stonebraker2020surveillance} for object detection models to account for on-demand video and image processing.
%%%%
% During our collaboration with local police department and MIT for building Missing-person Query engine \cite{solaiman2021applying} and Human-in-the-Loop Video Querying system \cite{stonebraker2020surveillance}, we solved this problem by building a multimodal knowledge querying system \cite{palacios2019wip} by using \textit{entity-centric higher-level semantic features} (such as objects, object types, physical relations, e.g., a person wearing blue shirt, time and place of  an event), distributed systems, and the functionalities of RDBMS.
% leveraging the capabilities of the feature extraction models and the functionalities of RDBMS.
We developed SKOD during our collaborations with West Lafayette police department, MIT, and CMU for building a Missing-person Query engine \cite{solaiman2021applying} and a Human-in-the-Loop Video Querying system \cite{stonebraker2020surveillance}. SKOD was demonstrated at Northrop Grumman TechFest in 2019. The project has been funded by Northrop Grumman for three consecutive years since 2019, renewing the funding every year.
    \item 
    \textbf{Data Integration and Relevance Learning.} % and Relevance Learning/ Matching
Data integration from various sources to answer queries over a single view of the data to users
% , is confronted with a multitude of heterogeneity issues. These problems arise 
suffers from heterogeneity issues
from differences in feature % data attributes
names that hold similar data \cite{solaiman2022femmir, solaiman2021applying}, annotation mismatch, and variations in data schema and types \cite{solaiman2022open}. As data volume increases and the necessity to share existing data intensifies, importance of data integration becomes more prevalent for multimodal data recommendation.
%
% To address the challenges of data integration in domain-specific applications, I propose three data integration approaches. 
% The first approach, EARS, delivers integrated query results over time using a mediation approach and schema mapping \cite{solaiman2021applying}. The second approach, FemmIR, learns co-ordinated graph representation of the data samples comprised of their semantic features to deliver approximate matches \cite{solaiman2022femmir}. The third approach, WesJeM, uses Contrastive Learning to embed data-objects and their semantic properties in a high-dimensional space using higher semantic features in a data sample as weak labels \cite{solaiman2022open}, allowing zero-shot similarity matching and data discovery of multimodal data in open-world environments.
% http://mlwiki.org/index.php/Mediator_(Data_Integration)
My first proposed approach for data integration, \textbf{EARS}, delivers integrated query results over time using a mediator approach along with Postgres triggers. A semantic mapping is employed between the mediated schema and the data sources \cite{solaiman2021applying} to query the limited properties-of-interest in real applications. % schema mapping
% solving the problem of scalability and quick throughput. 
The original query-by-example is translated into conjunctive queries among data sources and a SQL-Join on the task-specific features is performed at run-time to integrate all the relevant sources to the query example.
%
The second approach, \textbf{FemmIR}, learns a co-ordinated graph representation of the data samples from their semantic features to deliver a ranked list of the relevant data samples to user queries \cite{solaiman2022femmir}. I proposed a novel Edit distance metric, \textit{CED},  to measure the amount of difference between two data samples based on their semantic features. FemmIR learns an embedding function that maps the features extracted from the input data sample and the query example to a similarity score based on the multiplicative comparison of the Hierarchical Attributed Relational Graph representations (HARG) of the features.
% Streaming data over time can be used online to train the model.
%
The third approach, \textbf{WesJeM}, uses Contrastive Learning to embed data-samples and their semantic properties in a high-dimensional space using higher-level semantic features in a data sample as weak labels \cite{solaiman2022open}, allowing zero-shot similarity matching and data discovery of multimodal data in open-world environment.


    \item 
    \textbf{Dealing with Open-world Novelties:}
%
% Building intelligent AI systems entail adaptation to changing situations. Traditional AI systems are limited in handling unexpected events or `novelties' that are not part of their training data or well-defined environments. We developed techniques for novelty characterization and adaptation at various stages of AI lifecycle, including data integration, relevance learning, environment modeling, feature extraction, and inherent domain properties. These techniques were applied in three areas: novelty characterization and difficulty estimation, robust feature extraction with dataset augmentation, and intrinsic domain complexity estimation for distributed AI systems. 
% The research proposes empirical frameworks and measures to handle these challenges and improve the performance of AI systems in real-world applications.
%
To construct intelligent AI systems, it is necessary to adapt to evolving scenarios. However, conventional AI systems face restrictions when it comes to managing unexpected events or "novelties" that were not previsouly seen or modeled. We need to characterize, detect and adapt to novelties at various stages of the AI life cycle, such as data integration, relevance learning, environment modeling, feature extraction, and inherent domain properties. Novelty characterization and difficulty estimation is required in a plethora of AI systems ranging from multimodal information retrieval \cite{solaiman2022open} (distribution change and concept drift), dataset complexity \cite{solaiman2023domainComplexity}, to visual (object detection in video and image \cite{nesen2021dataset}) and planning domains (games \cite{solaiman2022measurement} and war).
% We have applied these techniques to three main areas, namely: novelty characterization and difficulty estimation, robust feature extraction with dataset augmentation, and intrinsic domain complexity estimation for distributed AI systems.
\end{enumerate*}

\medskip

% \heading{Others}
% Work on clustering and data analysis
% Denis work

% \heading{Application}
% How is my research important?
% \begin{enumerate}
%     \item Finding missing persons
%     \item Data Discovery
%     \item Data Completeness
%     \item Examples of Data Fusion: Use cases \href{https://proxet.com/blog/data-fusion-and-data-integration-best-practices/}{HERE}

%%% Application %%%%%
% Mishra,Balmukund, Deepak Garg, Pratik Narang, and Vipul Mishra."Drone -surveillance for search and rescue in natural disaster." Computer Communications 156(2020):1-10.  
My work has shown positive results on open societal problems that previously required large-scale human endeavor and computational resources, such as Missing Person Search \cite{solaiman2021applying}, Dataset Complexity Estimation \cite{solaiman2023domainComplexity}, Search and Rescue in Disasters \cite{solaiman2013avra}, and Medical Triage. The impacts of my work in academia span across diverse areas, including Multimodal Information Retrieval, Data Discovery, Building Robust AI Agents, and Data Completion, thereby creating a significant impact.

%%%%% Impact %%%%
My research on Novelty Adaptation and Situational Knowledge Delivery has the capability to \textbf{transform information retrieval for everyone by empowering AI systems }to promptly and precisely retrieve the information required by users, even in situations where the resources are minimal, data sources are in a state of constant change, and involve various modalities, interconnections, and predictive arguments.
This can have broad implications for a variety of applications with societal and user-end impacts, including contextual search engines, digital libraries, AI-assisted journaling and mindfulness, urban information system, etc.


% following can go to cover letter
% In summary, my research strives to enhance the effectiveness and efficiency of multimodal information retrieval in open-world environments, benefiting various applications like web search engines, e-commerce platforms, and digital libraries. To achieve this goal, I focus on the functionalities of cross-modal matching, situationally-aware data recommendation, and scalable data management systems, creating robust systems and interactive algorithms. Moreover, I develop scientific principles to quantify and characterize novelty in open-world domains, and scalable AI systems that can react to these novel events. I tackle numerous design challenges, such as resource-aware data management, data integration, and addressing open-world novelties, to build intelligent AI systems, including situational knowledge recommenders, that can deal with diverse data modalities and evolving information needs. Ultimately, my research contributes to advancing the development of AI systems that can handle the challenges of data-centric applications in open-world environments.
    
\medskip

% \customsection*{
\bfheading{Research Philosophy}
I believe that the ultimate objective of AI is not just to enhance performance on isolated tasks but to complement human abilities in solving persistent issues in real world, minimize human labor through responsible action, harness the power of data for our benefit, and make decisions that have long-lasting positive effects. My work is driven by a desire to integrate these ideas into a context that prioritizes social good and encourages the consumption and sharing of healthier information. This involves leveraging multimodal techniques to better understand complex situations, adapting to different situations, and providing trustworthy and easily understandable information for humans.

% Philosophy 1: Me
% I believe that AI should not be solely focused on improving performance on isolated tasks, but should also work to complement human abilities in solving real-world problems. In addition, AI should aim to minimize human labor through responsible action, harness the power of data for our benefit, and make decisions that have long-lasting positive effects. My work is driven by a desire to integrate these ideas into a context that prioritizes social good and encourages the sharing of healthier information. This involves using multimodal techniques to better understand complex situations, and delivering trustworthy and easily understandable information to users.

% Philosophy 2: % Li
% The primary purpose of AI should not be to achieve high scores on specific tasks, but rather to help humans perform tasks in the real world more effectively. AI can assist in reading, seeing, hearing, and decision-making, all of which have significant implications for society. To promote healthier information consumption and dissemination, my work is grounded in a social good context that emphasizes trustworthy, explainable knowledge presented through multimedia tools such as language and visual aids. By leveraging these techniques, I hope to contribute to a more informed and empowered society.
\medskip
% % ChatGPT
\customsection*{Resource Aware Data Management for Multimodal Applications}
%
In designing data management systems for modern applications, such as missing person search, disaster resource management, triage, and emotion recognition, the focus has shifted to account for data-at-rest and streaming input while ensuring scalability to handle increasing information needs and data ingestion. To address these challenges, 
%%%%%%
during our collaboration with local police department and MIT for building Missing-person Query engine \cite{solaiman2021applying} and Human-in-the-Loop Video Querying system \cite{stonebraker2020surveillance}, we proposed a novel multimodal knowledge querying system called \textbf{SKOD} (Situational Knowledge on Demand) \cite{palacios2019wip, stonebraker2020surveillance}. SKOD leverages entity-centric higher-level semantic concepts
%%%%
(such as objects, object types, physical relations, e.g., a person wearing blue shirt, time and place of  an incident),  
%%%
and the functionalities of distributed systems and RDBMS
%%%
to query domain-specific information needs in practical multimodal applications. SKOD was developed in collaboration with Northrop Grumman, MIT, and CMU and demonstrated at Northrop Grumman TechFest in 2019. The project has been funded by Northrop Grumman for three consecutive years since 2019, renewing the funding every year.


\semisection*{
    Heterogenous Data Ingestion, Scalability, and Delivery-on-demand}
    We used Postgres as the backend architecture  for both data storage and on-time delivery. 
    %%
    Building on top of the RDBMS allows us to scale to practical data % video 
    volumes, as well as using the querying interface with query-by-example and query-by-features dramatically lowers the human costs of 
    % video-driven investigations \cite{stonebraker2020surveillance}.
    multimodal and visual domain search \cite{stonebraker2020surveillance}.
    %%
    For consuming data from heterogeneous sources (both at rest and streaming), I integrated Kafka producers and consumers on top of SKOD \cite{palacios2019wip}. 
    %
    Any query to the system was formulated and considered as an \textit{incident in real life}.
    For delivery-on-demand from incomplete modalities, we used Postgres Trigger functionality, which 
    %
    is activated whenever an insert occurs that matches a certain incident (any matching data). 
    This feature allows us to deliver incomplete 
    information need %data 
    and complete it later when new matching data is encountered, while being capable of adapting to changing information requirements. Queries in SKOD can be both standing queries or one-shot queries. To deal with the changing requirements, we proposed to build a \textit{\textbf{query-drive knowledge base}} for each user, where all queries can relate to a single incident. SKOD speeds up the data delivery by storing frequent incidents by caching hot queries, and recently used data.


     \semisection*{
    Resource-constrained Feature Extraction}
    Task-specific querying systems face challenges during the data preparation stage due to low-quality data sets and a lack of labeled training samples. Although large-scale language models have made significant progress, there exists very few task-specific attribute extractors for text. Our team addressed these issues in SKOD by implementing a \textit{priority polling system} that selects candidate data samples for feature extraction from videos and images, instead of immediately processing features for batch inputs. This feature, coupled with trigger functionality, enables us to provide information needs on-demand and complete them with time. Additionally, we developed a \textbf{cloth-color extractor} for videos using common-sense reasoning and color and shape analysis \cite{stonebraker2020surveillance} on top of YOLO. To identify attributes in unstructured text, I propose a model called \textbf{HART} \cite{solaiman2022femmir}, which solves the problem in two stages: (i) \textbf{candidate sentence identification} by transforming the problem into a similarity-search problem using pre-trained language representation models (SBERT) and lexical knowledge bases, and (ii) \textbf{semantic attribute understanding} using syntactic characteristics and lexical meanings of the tokens in the candidate sentences. This approach can be generalized for any domain and lays the groundwork for intelligent document processing.
    % For unstructured text, I propose an attribute identification model, \textbf{HART} \cite{solaiman2022femmir} which solves the problem in two stages: \textbf{candidate sentence identification} with by forming the problem as a similarity-search problem using pre-trained language representation models (SBERT) and lexical knowledge bases, and \textbf{semantic attribute understanding} using syntactic characteristics and lexical meanings of the tokens in the Candidate Sentences. This approach can be generalized for any domain and  paves the path for intelligent document processing. 
\smallskip
\customsection*{Label-efficient Data Integration at Higher Semantic Level}

\semisection*{
View-based Data Integration.}
% Since information need for users in domain-specific applications are based on specific properties-of-interest, multimodal data needs to be integrated based on these properties. 
Traditional data integration approaches suffer because of heterogeneity among data sources and 
missing %incomplete 
modalities. Machine learning models for multimodal data fusion learn joint representations to exploit complementarity and redundancy of multiple modalities, but overlooks the information needs based on higher-level semantic concepts. With the use of Postgres trigger and by using a mediated schema for each queried incident, SKOD delivers integrated query results over time. Since the number of properties-of-interest are quite moderate in many emerging applications, using an approach similar to the \textit{Global-as-View (GAV)} data integration, I proposed to employ \textbf{schema mapping} between the \textit{mediated schema} from the query incident and \textit{local schema} from different data sources. The proposed approach, \textbf{EARS} \cite{solaiman2021applying} adopts an entity-relationship-attribute schema for each new data source, and a wrapper is designed to translate the source schemas to the mediated schema. The original queries are translated into conjunctive queries between features among data sources and a SQL-Join is performed at run-time to integrate all the relevant sources. Using the versatility of Postgres, we achieve the scalability and speed that is required for time sensitive use-cases, with minimal amount of computational resources. 
% Paper was published in IEEE Computer journal.

\semisection*{Approximate Matching using Graph Representation Learning. }
While the SQL-JOIN based relational DBMS approach allows a lot of flexibility, it does not utilize the historical knowledge of previous queries,
% (previous queries become training samples),
and cannot perform approximate matching. Considering the sensitivity of some open-world application domains, it is desirable to search for approximate relevance between different modalities and sources. 
Motivated by
% the 'limited properties-of-interest' property
representation-invariant properties of graph representation models and 
combined with the existing works on approximate graph matching techniques, 
% (look at the COLON paper to see graph/GNN advantage), 
I propose \textbf{co-ordinated graph representation learning of the data samples comprised of their semantic features} \cite{solaiman2022femmir}, where it learns to approximate a novel Edit distance metric, \textit{Content Edit Distance (CED)}, based on the multiplicative comparison of the \textit{Hierarchical Attributed Relational Graph} representations. 
% Paper got two positive review from SIGMOD 2023.
\smallskip
\semisection*{Weakly Supervised Metric Learning for Cross-Modal Matching}
For real-world systems, \textit{data discovery 
from heteregenous modalities}
%based on \textit{relevance matching on domain-specific objective function}
% higher-level semantic properties} 
and \textit{explanation of the relevant properties among similar data objects} is of equal importance. 
Since in these applications, manual annotation is not feasible or they lack annotation resources, we need alternative supervision techniques for cross-modal matching. 
%
Motivated by the advancement in translation and captioning models (video/audio $\rightarrow$ text), I propose to embed data-objects and their semantic properties in a high dimensional embedding space via Contrastive Learning. 
After extracting the interaction among entity-centric higher-level semantic features (such as, topics, events, entities, triplets) from texts and other translated modalities, a \textit{data information network} is built by connecting data samples to
their features via their interactions.
Finally, I construct a structure-infused representation for the data-objects from all modalities in \textbf{WesJeM} \cite{solaiman2022open}, by jointly embedding the data samples, the features, and the available similarity labels, 
% associated with them, 
in a single space.
%
For learning, I defined a multi-task learning objective capturing the interaction information,
% among data-objects and their features
by aligning the
representation of the data samples, defined by their textual content, with the representation of features, based on their common relations. 
% positive and negative samples are selceted by that method
%
For open-world environment where data and information-need keep changing, along with the dynamic data sources, WesJeM opens up the path for \textbf{Zero-Shot similarity matching} and \textbf{Data Discovery} of multimodal data.

\smallskip
\customsection*{Adaption to Open-world Novelties}

AI systems are often limited by their inability to handle unexpected events that are not part of their training data or well-defined environments. These significant changes or events are referred to as \textbf{`novelties'} under DARPA SAIL-ON project, and their characterization and adaptation is critical for real-world applications. To build robust and intelligent AI systems, I developed novelty characterization and adaptation techniques at various stages, including data integration and relevance learning, environment modeling, feature extraction, training, and domain or data level.

\semisection*{Novelty Characterization, Detection, and Difficulty Estimation}
I characterized the \textbf{novelties encountered in multimodal information retrieval} in \cite{solaiman2022open} and proposed how WesJeM can be adapted for changing data patterns and incomplete or noisy modalities in data integration and relevance learning stage.
Moreover, motivated by the information-theoretic approach for difficulty estimation of novelties, I proposed an empirical framework for novelty characterization and difficulty estimation in \textbf{planning domains} \cite{solaiman2022measurement}. For a reinforcement-learning based Monopoly agent, graphically modeling the environment to augment the state and action space allow to integrate graph edit distance as a novelty difficulty metric.
%
\semisection*{Robust Feature Extraction with Dataset Augmentation}
The efficiency of entity-centric machine learning models in response to novelties depends on the efforts during the model training, design and data collection stages. We proposed a \textbf{novelty generation framework} \cite{nesen2021dataset} at
the data preparation stage of training a model to assure its robustness and reduce the bias. We augmented the original dataset in a domain-agnostic
and budget efficient manner with generated novelties for visual modalities, %perception domain%
and improved the \textbf{novel object detection} performance with the augmented dataset.
%
%
\semisection*{Intrinsic Domain Complexity Estimation for Distributed AI Systems} 
Understanding of the inherent characteristics of the domain is essential for novelty characterization and model adaptability. % for building in fail-safe into the model
We proposed an \textbf{application-independent domain complexity} measure for the AI systems in perception domain \cite{solaiman2023domainComplexity} using \textbf{federated learning as the reference paradigm} to handle distributed dataset operations.
% We complement existing works in domain complexity estimation, by using inherent 
Building upon intrinsic dataset properties such as dimensionality, heterogeneity and sparsity for singular environment, we created a complexity metric \textit{for the distributed environment}, showing efficacy for classification task. 

% \medskip
\customsection*{Future Research Agenda}
Currently, we are experiencing a thrilling era for multimodal information processing and robust AI research since it is highly supported by the core programs in NSF's Division of Information and Intelligent Systems (IIS) and by "Harvesting the Data Revolution ($HDR^2$)" idea - second wave of one of the 10 big ideas
% supported
by NSF for long-term research.
% Division of Information and Intelligent Systems (IIS)

% Looking forward, I plan to continue investigating the interaction of data management systems with cloud computing, modern hardware (e.g., emerging storage and network), and novel applications (e.g., data science and IoT).

My \textbf{long-term goal} is to create intelligent systems that can reason, learn and cooperate with humans to improve the standard of living by utilizing the vast amounts of data available in the modern era. My focus is to devise new algorithms and methods that can make a significant impact on society, leverage existing scientific advancements, and address real-world challenges. To that end, I
plan to continue my research on \textit{multimodal data management in real world} by approaching from the following directions:


\semisection*{User Preference Modeling}
To complete the life-cycle of \textit{situational knowledge delivery}, we still have challenges in modeling user's information need in a robust and efficient manner in multiple directions \cite{solaiman2021applying}:
\begin{enumerate*}[label=(\arabic*)]
    \item user requirement is not always obvious or explicitly stated,
    \item user can be interested in multiple types of events and knowledge bases with varying probabilities,
    \item learning algorithms need to \textit{adapt to changing user preferences with time}.
\end{enumerate*}
%%%
% Multiple Data of Interest to Same User - user interested in multiple knowledge base.
 %%%%%
%%%
I aim to develop novel algorithms using techniques such as active learning and reinforcement learning that can accurately capture and predict users' preferences based on their behavior, interactions, and feedback.
Understanding the features that drive user preferences, and leveraging this knowledge to improve personalized recommendations and user experience, has applications in education (student advising, classroom teaching), e-commerce, healthcare, etc.
%
% Builds User Profiles using the history of user queries
% Active Learning to narrow/expand intention model with more interaction
% Expands user queries with word embedding models to fetch relevant data from the database.
%
To achieve this research goal, collaborations with researchers in \textbf{ human-computer interaction, psychology, and marketing} will be essential. 
%%
\semisection*{Explainability and Trustworthiness in Data Recommendation}
% and information consumption}
As the amount of multimodal data generated and consumed by users is increasing, there is a growing need for users to understand the basis for recommendations \cite{solaiman2022femmir}
% (similar features \cite{solaiman2022femmir}) 
and the saliency and trustworthiness of the information being consumed. This is especially important in sensitive domains such as \textit{healthcare, finance, and legal decision-making} to allow for tracking, cross-checking with social contexts and verification.
% My research aims to develop novel models and techniques that enable users to better understand and trust the data they receive.
To achieve this goal, collaboration across multiple areas is necessary, including\textbf{ data science, 
% machine learning,
natural language processing, computer vision,  human-computer interaction, and ethics}. 
% By bringing together experts from these areas, we can create more transparent and explainable models for data recommendation and consumption. Additionally, 
With this, we can ensure that these models are designed with the user in mind, taking into account their cognitive and perceptual abilities. This collaboration can also lead to the \textit{development of ethical guidelines and principles for designing trustworthy systems}, ensuring that users' rights and privacy are protected.
%%%
%%
\semisection*{Privacy preserving Data Dissemination and Federated Learning}
To address the growing concern over data privacy, particularly in medical and identity contexts, research in privacy-preserving multimodal data dissemination and federated learning is crucial, as identified in SKOD framework \cite{palacios2019wip}. Further research to integrate the use of local data processing and remote federation with multimodal machine learning techniques is needed to ensure this new requirement in information processing, while understanding and formalizing the resource requirements. 
% Federated learning offers a promising approach for training large multimodal models without the need to share data with other sites, ensuring data privacy while still benefiting from data from other sites. 
Collaboration across various fields such as \textbf{information security, statistics, data management, law, ethics, and public policy} is vital to advance research in this area.

%%%%%% Federated Learning / Data virtualization/data federation
% 5.11 and 5.12 in \href{https://arxiv.org/pdf/2303.06471.pdf}{``mmdintegration for oncology''}.

% \semisection*{Data Democratization}
    % Missing piece of information in one modality can be filled in with similar information from another modality.
    % fill in missing data one modality with another

\semisection*{Information Completion and Data Democratization}
As data becomes increasingly important in all domains, there is a need for new techniques that enable individuals and organizations to efficiently extract insights from data and complete missing information. To address this challenge, future research should focus on developing advanced machine learning models that are able to perform well even with incomplete data, as well as methods for effective data integration and knowledge transfer within organizations. Collaboration is needed between \textbf{machine learning experts, data management specialists, and domain experts in various fields} to achieve a comprehensive and effective solution for data democratization and information completion.


% With advant of IIS, CISE-IIS (info-integration-informatics)https://new.nsf.gov/funding/opportunities/iis-information-integration-informatics-iii
% and IIS-Human-Centered Computing, IIS-Robust Intelligence (RI), and Safe Learning-Enabled Systems
% https://www.nsf.gov/news/special_reports/big_ideas/nsf2026.jsp
% Aidan Zhang: https://www.nsf.gov/awardsearch/showAward?AWD_ID=2008208&HistoricalAwards=false
\medskip
\customsection*{Collaboration and Funding}
My future research vision requires collaboration with expert researchers in many
fields, including natural language processing, computer vision, machine learning, data mining, social science, human computer interaction, systems and databases. 
%%
I gained extensive expertise in overseeing and directing major projects, encompassing teams of over 12 individuals and collaborating with various universities and institutions.
I led multiple masters and undergraduate students, collaborated with multiple Ph.D. students and coordinated with 5 professors from different universities to participate in the REALM project. I am fortunate to have close collaborations with professors from multiple universities and research institutes, such as Massachusetts Institute of Technology (MIT), University of Michigan (UMichigan), University of Southern California (USC), Information Sciences Institute (ISI), Institute for Defense Analyses (IDA), University of Massachusetts (UMass), Middle East Technical University (METU), etc. I also have had the fortune to work closely with researchers from databases and applications, along with end-users and program managers to conduct interdisciplinary research. 
% I plan to continue existing collaborations and foster new connections in order to develop well-established principles underlying multimodal knowledge and novelty in learning models research.
I intend to maintain my current collaborations while actively cultivating new partnerships to advance the establishment of robust principles that underpin research in multimodal knowledge and novelty in learning models.

During my Ph.D., my work has been mainly supported by the Northrop Grumman Corporation (NGC), 
% Research Consortium for Artificial Intelligence and Machine Learning
DARPA, ARFL, and Sandia National Lab. 
% I have been nominated by DARPA as a DARPA Riser (awarded to young faculties, postdocs, and senior PhDs who will lead to technological breakthroughs), and I presented ideas directly to DARPA project managers during the invitation-only DARPA Forward Event.
Additionally, I have contributed significantly to the writing of grant proposals, including idea
generation, method design, idea illustration and visual aid creation, such as 
% DARPA KAIROS project,
DARPA ITM project, and DARPA Triage Challenge.
% DARPA SemaFor project, DARPA CCU project, and NSF MMLI project. 
As a future faculty, I will continue to seek funding opportunities in the future from early career supports, various funding agencies (e.g., DARPA, ARL, AFRL, IARPA, NSF, NIH, DOE, DOD) and industries (e.g., NGC, Microsoft, IBM, Ford, Meta, Google, Intel).
Specifically I will aim for NSF CAREER, NSF CRII, NSF EAGER, NSF ADVANCE, OSR young investigator programs from DOE, DARPA, AFRL, NASA, and other research awards.


% CISE .nsf.gov

% Ford alliance

% Sandia labs

% AFRL norm Ahmed mark linderman

% NGC Jason Kobe’s

% PLM manufacturing center

% Career award training at Purdue

% Talk some ML and AI for security or other directions


% \section{Publications}

$*$ Equal Contributions;   $\mp$ Presentations; $\dagger$ Preprints; 
%% Specify your last name(s) and first name(s) as given in the .bib to automatically bold your own name in the publications list. 
%% One caveat: You need to write \bibnamedelima where there's a space in your name for this to work properly; or write \bibnamedelimi if you use initials in the .bib
%% You can specify multiple names, especially if you have changed your name or if you need to highlight multiple authors. 
\mynames{
Solaiman$*$/KMA,
Solaiman/KMA,
}
%% MAKE SURE THERE IS NO SPACE AFTER THE FINAL NAME IN YOUR \mynames LIST

\nocite{*}

\smallskip

% \printbibliography[heading=pubtype,title={\printinfo{\faFile*[regular]}{Journal Articles}},type=article]

% \divider

% \printbibliography[heading=pubtype,title={\printinfo{\faUsers}{Conference Proceedings}},type=inproceedings]

\printbibliography[heading=pubtype, type=article]
\printbibliography[heading=pubtype, type=inproceedings]

\bibliographystyle{plain}
\bibliography{ref}

\end{document}

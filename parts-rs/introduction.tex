% \bfheading{Research Vision and Research Goal}

Real-world use-cases in data-centric aplications 
%% adding from detailed
(including societal, healthcare, or education) with minimal computational resources % ending add
often have an unprecedented influx of unstructured and noisy data from multiple sources and modalities. 
Extraction of meaningful information % refers to recommendation
from such 
% overwhelming and dirty (heteregenous) 
heteregenous and changing % refers to mmir and novelty
datasets requires achieving the complementary 
% goals of
functionalities of 
% finding solutions for the challenges of 
% computational efficiency and contextually-aware multimodal information recommendation.
\textit{
cross-modal matching,
% multimodal information (/data) retrieval,
%%%%
% adaptable learning models, % and systems (for the intric dataset complexity work)
scalable data management system (mostly search engines and databases),
%%%%
}and \textit{situationally-aware data recommendation.}
%%%%% Same data can be interpreted differently by different users based on the context.
%%%%%%%%%%%%%%%%%%%%%%%%%%%%%%%%%%%%%%%%%%%%
% , data management, data integration and relevance, situational awareness, and user modeling.
My research goal is to understand how these
functionalities can be achieved by % goals -> functionalities
designing interactive algorithms and 
robust systems %techniques 
that solve the problem of 
\textbf{situational knowledge on demand in open-world}
% on-demand multimodal information recommendation % retrieval 
from multiple forms of information, regardless of whether
presented as text, images, videos, audio, or other modalities,
while achieving data-driven and resource-aware \textit{data management} and \textit{data integration} capabilites. 
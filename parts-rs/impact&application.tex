
% \heading{Others}
% Work on clustering and data analysis
% Denis work

% \heading{Application}
% How is my research important?
% \begin{enumerate}
%     \item Finding missing persons
%     \item Data Discovery
%     \item Data Completeness
%     \item Examples of Data Fusion: Use cases \href{https://proxet.com/blog/data-fusion-and-data-integration-best-practices/}{HERE}

%%% Application %%%%%
% Mishra,Balmukund, Deepak Garg, Pratik Narang, and Vipul Mishra."Drone -surveillance for search and rescue in natural disaster." Computer Communications 156(2020):1-10.  
My work has shown positive results on open societal problems that previously required large-scale human endeavor and computational resources, such as Missing Person Search \cite{solaiman2021applying}, Dataset Complexity Estimation \cite{solaiman2023domainComplexity}, Search and Rescue in Disasters \cite{solaiman2013avra}, and Medical Triage. The impacts of my work in academia span across diverse areas, including Multimodal Information Retrieval, Data Discovery, Building Robust AI Agents, and Data Completion, thereby creating a significant impact.

%%%%% Impact %%%%
% The significance of my research lies in its potential to improve the efficiency and effectiveness of multimodal information retrieval in open world environments. By developing new techniques for multimodal information retrieval, my research can help users to quickly and accurately find the information they need, even in situations where the data sources are constantly changing. 
%%%%
% My work on Situational Knowledge Delivery and Novelty Adaptation has the potential to revolutionize information retrieval for everyone by enabling AI systems to quickly and accurately find the information user need, even in situations where the data sources are constantly changing, and involve multiple modalities, interconnections, and predictive arguments. 
% Rewritten as follows:
My research on Novelty Adaptation and Situational Knowledge Delivery has the capability to \textbf{transform information retrieval for everyone by empowering AI systems }to promptly and precisely retrieve the information required by users, even in situations where the resources are minimal, data sources are in a state of constant change, and involve various modalities, interconnections, and predictive arguments.
% This can have broad implications for a variety of applications with societal and user-end impacts, including contextual search engines, digital libraries, AI-assisted journaling and mindfulness, urban information system, etc.


% following can go to cover letter
% In summary, my research strives to enhance the effectiveness and efficiency of multimodal information retrieval in open-world environments, benefiting various applications like web search engines, e-commerce platforms, and digital libraries. To achieve this goal, I focus on the functionalities of cross-modal matching, situationally-aware data recommendation, and scalable data management systems, creating robust systems and interactive algorithms. Moreover, I develop scientific principles to quantify and characterize novelty in open-world domains, and scalable AI systems that can react to these novel events. I tackle numerous design challenges, such as resource-aware data management, data integration, and addressing open-world novelties, to build intelligent AI systems, including situational knowledge recommenders, that can deal with diverse data modalities and evolving information needs. Ultimately, my research contributes to advancing the development of AI systems that can handle the challenges of data-centric applications in open-world environments.
    
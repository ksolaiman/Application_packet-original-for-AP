
User satisfaction for multimodal information extraction in real-world depends on few pivotal aspects:
\textcolor{red}{Challenges:}
\begin{enumerate*}[label=(\roman*)]
    \item Real-world datasets coming from multiple independent sources and modalities are often heterogeneous, noisy, and incomplete. % and changes with time 
    A desirable system needs to consume and process large number of streaming and at-rest heterogeneous data on demand.
    % Missing piece of information in one modality can be filled in with similar information from another modality.
    % fill in missing data one modality with another
    %
    \item 
    % Data has to be delivered on time. 
    On-demand use-cases require online data delivery and slightest delay can render the delivered data useless.
    % Quick delivery time/ How is data delivered on demand
    %
    \item Most use cases in open-world lack the class labels and require automated integration of data sources. % suffer from \textit{lack of labels}.
    % \item The amount of multimedia data is large, and it streams from multiple sources. %%%%%%
    \item With the influx of new data, it is necessary to avoid re-inventing information extraction models. Retrieval algorithms need to use the existing feature extraction models, which in turn needs to adapt to the resource constraints in open-world systems.
    % RESOURCE Constrainted Feature Extraction
    \item Data retrieval systems need to assess  the intrinsic complexity of the datasets and has to be adaptable to novel changes in data (i.e., data shift and concept drift).
    %%%%%%%%% Rest can be put into future work
    % \item User requirement is not always obvious or explicitly stated. Learning algorithms need to adapt to changing user preferrence and delivered data must be relevant to user requirement. %%%%%
    % \item Without proper context and explanation the delivered data is not useful to user.  %%%%%
    % % Explanability
    %%%%%% Federated Learning / Data virtualization/data federation
    %%%%%% Data Democratization
    % Missing piece of information in one modality can be filled in with similar information from another modality.
    % fill in missing data one modality with another
\end{enumerate*}

\textit{In particular, I will focus on developing new methods for incorporating contextual information and user feedback to enhance the relevance and accuracy of search results.}

% \textcolor{red}{
% \begin{enumerate}[label=(\roman*)]
%     % \item Resource constrained Data-preprocessing (includes feature extraction - SurvQ, HART \cite{stonebraker2020surveillance, solaiman2022femmir}) 
%     \item Heteregenous (Real-time streaming and at-rest) data consumption, Queries model the knowledge,  Resource constrained feature extraction - SurvQ, HART \cite{stonebraker2020surveillance, solaiman2022femmir}) 
%     \item Scalability and On-time Data delivery (\cite{palacios2019wip, stonebraker2020surveillance})
%     \item Label-independent Data Integration
%     \begin{enumerate}
%         \item Focused on speed and scale \cite{solaiman2021applying}
%         \item Weakly supervised with approximate matching and adaptability to novel data sources \cite{solaiman2022femmir, solaiman2022open}
%     \end{enumerate}
%     % \item Dataset and Novelty Analysis
%     % \begin{enumerate}
%     %     \item Intrinsic Complexity of Datasets \cite{solaiman2023domainComplexity}
%     %     \item Dataset Augmentation with generated novelty
%     %     \item Novelty Characterization \cite{solaiman2022measurement}
%     % \end{enumerate}
% \end{enumerate}
% }


% FIGURE
% \heading{Scalability to videos of pratical volume: Query outputs and delivery-on-demand}\\
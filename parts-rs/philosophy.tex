
% \customsection*{
\bfheading{Research Philosophy}
I believe that the ultimate objective of AI is not just to enhance performance on isolated tasks but to complement human abilities in solving persistent issues in real world, minimize human labor through responsible action, harness the power of data for our benefit, and make decisions that have long-lasting positive effects. My work is driven by a desire to integrate these ideas into a context that prioritizes social good and encourages the consumption and sharing of healthier information. This involves leveraging multimodal techniques to better understand complex situations, adapting to different situations, and providing trustworthy and easily understandable information for humans.

% Philosophy 1: Me
% I believe that AI should not be solely focused on improving performance on isolated tasks, but should also work to complement human abilities in solving real-world problems. In addition, AI should aim to minimize human labor through responsible action, harness the power of data for our benefit, and make decisions that have long-lasting positive effects. My work is driven by a desire to integrate these ideas into a context that prioritizes social good and encourages the sharing of healthier information. This involves using multimodal techniques to better understand complex situations, and delivering trustworthy and easily understandable information to users.

% Philosophy 2: % Li
% The primary purpose of AI should not be to achieve high scores on specific tasks, but rather to help humans perform tasks in the real world more effectively. AI can assist in reading, seeing, hearing, and decision-making, all of which have significant implications for society. To promote healthier information consumption and dissemination, my work is grounded in a social good context that emphasizes trustworthy, explainable knowledge presented through multimedia tools such as language and visual aids. By leveraging these techniques, I hope to contribute to a more informed and empowered society.
\textbf{Resource-aware Data Management:}
% Obtaining optimal performance in open-world multimodal applications requires designing Data Management System that can handle domain-specific workloads, low computational resources, and changing requirements. 
% During our collaboration with local police department and MIT for building Missing-person Query engine \cite{solaiman2021applying} and Human-in-the-Loop Video Querying system \cite{stonebraker2020surveillance}, we built a multimodal knowledge querying system \cite{palacios2019wip} by leveraging the capabilities of the feature extraction models and the functionalities of RDBMS.
% %%
% % higher-level 
% For entity-cetric semantic concept extraction from unstructured text, I proposed a candidate sentence identification and semantic attribute understanding model from unstructured texts \cite{solaiman2022femmir}.
% %%
% For video and image, we proposed priority-polling \cite{stonebraker2020surveillance} for large scale object and attribute detection models in video and images.
% % Will not mention color-sampling in intro-detail, but will mention in long detail. 
%
%
%%%%%%%%%%%%%%%%%%%%%%%%%
%%%%%%%%%% Junagaou style %%%%%%%%
Data management systems require a comprehensive understanding of the data properties, user requirements, and limitations imposed by open-world to achieve optimal performance and scalability 
for multimodal data recommendation. 
Emerging multimodal applications impose challenges for traditional system (hardware and software) design choices ranging all the way from input (heterogeneous sources, context and modalities) and output (delivery-on-time, quick throughput and diverse users), changing information needs (knowledge base creation and query), to computational resources (lack of annotations, domain-specific feature extractors or human resources). 
For instance, we showed that transfer learning for fine-grained semantic concept extraction from videos turned out to be ill-suited in large-scale systems \cite{stonebraker2020surveillance}.
%%%%
% During our collaboration with local police department and MIT for building Missing-person Query engine \cite{solaiman2021applying} and Human-in-the-Loop Video Querying system \cite{stonebraker2020surveillance}, we solved this problem by building a multimodal knowledge querying system \cite{palacios2019wip} by using \textit{entity-centric higher-level semantic features} (such as objects, object types, physical relations, e.g., a person wearing blue shirt, time and place of  an event), distributed systems, and the functionalities of RDBMS.
% leveraging the capabilities of the feature extraction models and the functionalities of RDBMS.
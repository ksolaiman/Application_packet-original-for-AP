\textbf{Dealing with Open-world Novelties:}
%
% Building intelligent AI systems entail adaptation to changing situations. Traditional AI systems are limited in handling unexpected events or `novelties' that are not part of their training data or well-defined environments. We developed techniques for novelty characterization and adaptation at various stages of AI lifecycle, including data integration, relevance learning, environment modeling, feature extraction, and inherent domain properties. These techniques were applied in three areas: novelty characterization and difficulty estimation, robust feature extraction with dataset augmentation, and intrinsic domain complexity estimation for distributed AI systems. 
% The research proposes empirical frameworks and measures to handle these challenges and improve the performance of AI systems in real-world applications.
%
To construct intelligent AI systems, it is necessary to adapt to evolving scenarios. However, conventional AI systems face restrictions when it comes to managing unexpected events or "novelties" that were not previsouly seen or modeled. We need to characterize, detect and adapt to novelties at various stages of the AI life cycle, such as data integration, relevance learning, environment modeling, feature extraction, and inherent domain properties. Novelty characterization and difficulty estimation is required in a plethora of AI systems ranging from multimodal information retrieval \cite{solaiman2022open} (distribution change and concept drift), dataset complexity \cite{solaiman2023domainComplexity}, to visual (object detection in video and image \cite{nesen2021dataset}) and planning domains (games \cite{solaiman2022measurement} and war).
% We have applied these techniques to three main areas, namely: novelty characterization and difficulty estimation, robust feature extraction with dataset augmentation, and intrinsic domain complexity estimation for distributed AI systems.